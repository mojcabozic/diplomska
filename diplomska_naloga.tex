\documentclass[mat1, tisk]{fmfdelo}
% \documentclass[fin1, tisk]{fmfdelo}
% Če pobrišete možnost tisk, bodo povezave obarvane,
% na začetku pa ne bo praznih strani po naslovu, …

%%%%%%%%%%%%%%%%%%%%%%%%%%%%%%%%%%%%%%%%%%%%%%%%%%%%%%%%%%%%%%%%%%%%%%%%%%%%%%%
% METAPODATKI
%%%%%%%%%%%%%%%%%%%%%%%%%%%%%%%%%%%%%%%%%%%%%%%%%%%%%%%%%%%%%%%%%%%%%%%%%%%%%%%

% - vaše ime
\avtor{Mojca Božič}

% - naslov dela v slovenščini
\naslov{Kvaternioni in kvaternionske matrike}

% - naslov dela v angleščini
\title{Quaternions and matrices of quaternions}

% - ime mentorja/mentorice s polnim nazivom:
%   - doc.~dr.~Ime Priimek
%   - izr.~prof.~dr.~Ime Priimek
%   - prof.~dr.~Ime Priimek
%   za druge variante uporabite ustrezne ukaze
\mentor{doc.\ dr.\ Lucijan Plevnik}
% \somentor{...}
% \mentorica{...}
% \somentorica{...}
% \mentorja{...}{...}
% \somentorja{...}{...}
% \mentorici{...}{...}
% \somentorici{...}{...}

% - leto diplome
\letnica{2025} 

% - povzetek v slovenščini
%   V povzetku na kratko opišite vsebinske rezultate dela. Sem ne sodi razlaga
%   organizacije dela, torej v katerem razdelku je kaj, pač pa le opis vsebine.
\povzetek{Kvaternioni so zaradi svoje posebne strukture in lastnosti uporabno orodje v kvantni fiziki, robotiki, sodobnem računalništvu ipd. 
Obseg kvaternionov je primer nekomutativnega obsega, zato nas študij kvaternionskih matrik pripelje do zanimivih zaključkov. Hitro ugotovimo, da zaradi nekomutativnosti
marsikatera lastnost kvaternionskih matrik, ki se na prvi pogled zdi očitna, zahteva bolj poglobljen razmislek. V delu bomo obravnavali osnovne definicije in lastnosti kvaternionov in kvaternionskih matrik ter jih primerjali 
s kompleksnimi matrikami. Spoznali bomo alternativno definicijo kvaternionov in njeno uporabo pri študiju lastnosti matrik, dotaknili pa se bomo tudi študija lastnih vrednosti teh matrik.}

% - povzetek v angleščini
\abstract{Quaternions, due to their unique structure and properties, are a useful tool in quantum physics, robotics, modern computer science, and more. Quaternions are an example of a
non-commutative division ring. This leads to interesting insights in the study of quaternionic matrices. A quick observation shows that, due to noncommutativity, many properties of quaternionic 
matrices that at first glance seem obvious actually require deeper consideration. In this work, we will examine the basic definitions and properties of quaternions and quaternionic matrices 
and compare them with complex matrices. We will explore an alternative definition of quaternions and its application in the study of matrix properties, and also touch upon the study of 
eigenvalues of these matrices.}

% - klasifikacijske oznake, ločene z vejicami
%   Oznake, ki opisujejo področje dela, so dostopne na strani https://www.ams.org/msc/
\klasifikacija{15B33, 20G20}

% - ključne besede, ki nastopajo v delu, ločene s \sep
\kljucnebesede{kvaternioni\sep kvaternionske matrike\sep lastne vrednosti\sep kompleksna adjungirana matrika}

% - angleški prevod ključnih besed
\keywords{quaternions\sep quaternionic matrices\sep eigenvalues\sep complex adjoint matrix} % angleški prevod ključnih besed

% - angleško-slovenski slovar strokovnih izrazov
%\slovar{
% \geslo{angleški izraz}{slovenski izraz}
% ...
%}

% - ime datoteke z viri (vključno s končnico .bib), če uporabljate BibTeX
\literatura{literatura.bib}

%%%%%%%%%%%%%%%%%%%%%%%%%%%%%%%%%%%%%%%%%%%%%%%%%%%%%%%%%%%%%%%%%%%%%%%%%%%%%%%
% DODATNE DEFINICIJE
%%%%%%%%%%%%%%%%%%%%%%%%%%%%%%%%%%%%%%%%%%%%%%%%%%%%%%%%%%%%%%%%%%%%%%%%%%%%%%%

% naložite dodatne pakete, ki jih potrebujete
%\usepackage[utf8x]{inputenc} 

\usepackage{amssymb}
\usepackage{amsthm}
\usepackage{amsmath}
\usepackage{amsthm}
\usepackage{ulem}
\usepackage{physics}
\usepackage{enumitem}
\usepackage{mathrsfs}
\usepackage{mathtools}
\usepackage{siunitx}

\raggedbottom

\usepackage[%
    left = \glqq,%
    right = \grqq,%
    leftsub = \glq,%
    rightsub = \grq%
]{dirtytalk}

\usepackage[math]{cellspace}
\setlength\cellspacetoplimit{3pt}
\setlength\cellspacebottomlimit{3pt}

\expandafter\def\expandafter\normalsize\expandafter{%
    \normalsize%
    \setlength\abovedisplayskip{15pt}%
    \setlength\belowdisplayskip{15pt}%
    \setlength\abovedisplayshortskip{0pt}%
    \setlength\belowdisplayshortskip{10pt}%
}


\numberwithin{equation}{section}

% deklarirajte vse matematične operatorje, da jih bo LaTeX pravilno stavil
% \DeclareMathOperator{\...}{...}

% vstavite svoje definicije ...
% \newcommand{\...}{...}

\renewcommand\qedsymbol{{\footnotesize QED}}
\renewcommand{\theenumi}{\roman{enumi}}
\renewcommand{\labelenumi}{\theenumi.}

%%%%%%%%%%%%%%%%%%%%%%%%%%%%%%%%%%%%%%%%%%%%%%%%%%%%%%%%%%%%%%%%%%%%%%%%%%%%%%%
% ZAČETEK VSEBINE
%%%%%%%%%%%%%%%%%%%%%%%%%%%%%%%%%%%%%%%%%%%%%%%%%%%%%%%%%%%%%%%%%%%%%%%%%%%%%%%

\begin{document}

\nocite{zhang}


\section{Uvod}
Kvaternione je prvi opisal matematik William Rowan Hamilton leta 1843. Takrat so matematiki nanje gledali le kot na
zanimivo algebraično strukturo, v zadnjih letih pa so kvaternioni postali nepogrešljivo orodje v vedno širšem naboru 
področij. Kot prvi so v začetku 19.\@ stoletja njihovo strukturo začeli uporabljati kvantni fiziki, danes pa se kvaternione
uporablja tudi v računalniški grafiki, robotiki in na področju računalnškega vida. Kvaternioni v smislu algebraične 
stukture so zanimivi, saj so primer nekomutativnega obsega s posebno ciklično strukturo. \\


\noindent
O matrikah nad poljem znamo povedati marsikaj. Poznamo njihove lastne vrednosti, znamo izračunati njihove determinante, vemo,
kdaj se jih da diagonalizirati itd. Ko želimo te lastnosti študirati na kvaternionskih matrikah, hitro naletimo na oviro, 
saj elementi matrik ne komutirajo. Začnemo se spraševati, ali imajo lastne vrednosti kvaternionskih matrik podobne lastnosti kot lastne 
vrednosti kompleksne matrike. A morda to velja le za desne lastne vrednosti? V tem diplomskem delu si bomo skušali odgovoriti na ta in mnoga druga 
vprašanja. Poskusili bomo ugotoviti, v katerih pogledih so si kompleksne in kvaternionske matrike podobne, in za katere lastnosti
je komutativnost ključnega pomena. Prvi del diplomske naloge bomo posvetili osnovnim definicijam v obsegu kvaternionov in njihovim 
lastnostim. Sledil bo pregled definicij in lastnosti kvaternionskih matrik, kjer bomo spoznali definicijo in pomen kompleksne 
adjungirane matrike, zadnji del naloge pa bomo posvetili študiju lastnih vrednosti kvaternionskih matrik.

\newpage

\section{Kvaternioni}
\subsection{Osnovne definicije}

Najprej si oglejmo nekaj osnovnih definicij.

\begin{definicija}
    \textbf{Kvaternion} je vektor oblike 
        \[ x = x_{0} + x_{1}i + x_{2}j + x_{3}k, \; \; x_{0}, \, x_{1}, \, x_{2}, \, x_{3} \in \mathbb{R},\] 
        pri čemer velja
        $$i^2 = j^2 = k^2 = ijk = -1,$$
        $$ij = -ji = k,$$
        $$jk = -kj = i,$$
        $$ki = -ik = j.$$
        Taki vektorji tvorijo štirirazsežen vektorski prostor nad $\mathbb{R}$ z bazo $\{1, i, j, k\}$, ki ga imenujemo \textbf{kvaternioni} in ga označimo s $\mathbb{H}$.
\end{definicija}

\begin{definicija}
    Naj bosta $x = x_{0} + x_{1}i + x_{2}j + x_{3}k, \: y = y_{0} + y_{1}i + y_{2}j + y_{3}k \in \mathbb{H}$. V $\mathbb{H}$ defniriramo
    seštevanje in množenje na naslednji način:
    \begin{equation*}
        \setlength{\jot}{10pt}
            \begin{aligned}
                x + y &= (x_{0} + y_{0}) + (x_{1} + y_{1})  i + (x_{2} + y_{2})  j + (x_{3} + y_{3})  k \\
                xy &= (x_{0}y_{0} - x_{1}y_{1} - x_{2}y_{2} - x_{3}y_{3})
                    + (x_{0}y_{1} + x_{1}y_{0} + x_{2}y_{3} - x_{3}y_{2})i \\
                    & \quad + (x_{0}y_{2} - x_{1}y_{3} + x_{2}y_{0} + x_{3}y_{1})j
                            + (x_{0}y_{3} + x_{1}y_{2} - x_{2}y_{1} + x_{3}y_{0})k
            \end{aligned}
    \end{equation*}
\end{definicija}

\begin{definicija}
    Naj bo $x = x_{0} + x_{1}i + x_{2}j + x_{3}k \in \mathbb{H}$.
    Število
    \[x_{0} = \mathrm{Re} \: x \] 
    imenujemo \textbf{realni del} $x$,
    število
    \[ x_{1} + x_{2}i = \mathrm{Co} \: x \]
    imenujemo \textbf{kompleksni del} $x$,
    število
    \[ x_{1}i + x_{2}j + x_{3}k = \mathrm{Im} \: x \]
    pa imenujemo \textbf{imaginarni del} $x$.
    \textbf{Konjugiran} $x$ definiramo kot 
    \[ x^{*} = x_{0} - x_{1}i - x_{2}j - x_{3}k, \]
    \textbf{absolutno vrednost} $x$ pa definiramo kot 
    \[ |x| = \sqrt{x_{0}^2 + x_{1}^2 + x_{2}^2 + x_{3}^2} \, . \]
\end{definicija}

\subsection{Lastnosti kvaternionov}

V tem razdelku si bomo ogledali nekaj lastnosti kvaternionov.

\begin{trditev}\label{eq1}
    V $\mathbb{H}$ veljajo naslednje izjave.
    \begin{enumerate}
        \item Naj bo $a \in \mathbb{H}$. Enačba $\, ax = xa$ velja za vsak $x \in \mathbb{H}$ natanko tedaj, ko je $a \in \mathbb{R}$.
        \item Za vsak $x \in \mathbb{H}$ velja $x^2 = {(\mathrm{Re} \: x)}^2 - {|\mathrm{Im} \: x|}^2 + 2 \, \mathrm{Re} \: x \, \mathrm{Im} \: x$. 
        \item Za vsak $x \in \mathbb{H}$ velja $x^{*} x = x x^{*} = {|x|}^2$.
        \item Za vsaka $x, y \in \mathbb{H}$ velja $(xy)^{*} = {y}^{*} {x}^{*}$.
        \item Enakost $x^{*} = x$ velja natanko tedaj, ko je $x \in \mathbb{R}$.
        \item Enačba $x^2 = -1$ ima neskončno mnogo rešitev v $\mathbb{H}$.
        \item Vsak kvaternion $x$ lahko enolično zapišemo kot $x = c+ dj$ za $c, d \in \mathbb{C}$.
        \item Za vsak $c \in \mathbb{C}$ velja $j c = \overline{c} j$.
        \item Za vsaka $x, y \in \mathbb{H}$ velja $|xy| = |x| \, |y|$.
    \end{enumerate}
\end{trditev}

\begin{dokaz}
    Naj bodo $x, y, z \in \mathbb{H}$.
    \begin{enumerate}
        \item Naj bo $a = a_{0} + a_{1}i + a_{2}j + a_{3}k \in \mathbb{H}$, kjer $a_i \in \mathbb{R}$ za $i = 0, 1, 2, 3$, in naj za vsak $x \in \mathbb{H}$ velja
        $ax = xa$. Ker $a$ komutira z $i$, velja $ai=ia$ oziroma 
        $$(a_{0} + a_{1}i + a_{2}j + a_{3}k)i = i(a_{0} + a_{1}i + a_{2}j + a_{3}k).$$
        Obe strani razpišemo in dobimo
        $$-a_{1} + a_{0}i + a_{3}j - a_{2}k = -a_{1} + a_{0}i - a_{3}j  + a_{2}k.$$
        Sledi $a_{2} = a_{3} = 0$.
        Sedaj upoštevamo, da $a$ komutira z $j$, in dobimo
        $$(a_{0} + a_{1}i)j = j(a_{0} + a_{1}i).$$
        Iz tega podobno kot zgoraj sledi $a_{1} = 0$.
        Dobimo $a = a_0$ oziroma $a \in \mathbb{R}$. Obratno, če je $a\in\mathbb{R}$, potem kot skalar komutira z vsakim $x\in\mathbb{H}$, torej $ax=xa$.
        \item Pišimo $x = x_{0} + x_{1}i + x_{2}j + x_{3}k = \mathrm{Re} \: x + \mathrm{Im} \: x$. Oglejmo si 
        \begin{equation*}
        \setlength{\jot}{10pt}
            \begin{aligned}
                x^2 &= (\mathrm{Re} \: x + \mathrm{Im} \: x)(\mathrm{Re} \: x + \mathrm{Im} \: x) \\
                &= (\mathrm{Re} \: x)^2 + (\mathrm{Re} \: x)(\mathrm{Im} \: x) + (\mathrm{Im} \: x)(\mathrm{Re} \: x) + (\mathrm{Im} \: x)(\mathrm{Im} \: x)
            \end{aligned}
        \end{equation*}
        Po točki \textit{i}.\@ je to enako
        $$x^2 = (\mathrm{Re} \: x)^2 + 2 \,\mathrm{Re} \: x \, \mathrm{Im} \: x + (\mathrm{Im} \: x)^2.$$
        Sedaj izračunamo
        \begin{equation*}
        \setlength{\jot}{10pt}
            \begin{aligned}
                (\mathrm{Im} \: x)^2 &=  (x_{1}i + x_{2}j + x_{3}k)(x_{1}i + x_{2}j + x_{3}k)\\
                & =  x_{1}i^2 + x_{1}x_{2}ij + x_{1}x_{3}ik + x_{1}x_{2}ji + x_{2}j^2 \\
                &\quad + x_{2}x_{3}jk + x_{1}x_{3}ki + x_{2}x_{3}kj + x_{3}k^2 \\
                &= -x_{1}^2 -x_{2}^2 -x_{2}^2 \\
                &= -{|\mathrm{Im} \: x|}^2.
            \end{aligned}
        \end{equation*}
        Pri tem smo upoštevali enakosti $i^2 = j^2 = k^2 = -1$, $ij = -ji$, $jk = -kj$, $ki = -ik$ in točko \textit{i}. Sledi
        \begin{equation*}
            x^2 = (\mathrm{Re} \: x)^2 + 2 \,\mathrm{Re} \: x \, \mathrm{Im} \: x -{|\mathrm{Im} \: x|}^2.
        \end{equation*}
        \item Pišimo $x = \mathrm{Re} \: x + \mathrm{Im} \: x$ in ${x}^* = \mathrm{Re} \: x - \mathrm{Im} \: x$. Ker je $\mathrm{Re} \: x \in \mathbb{R}$, po točki \textit{i}.\@ 
        komutira z $\mathrm{Im} \: x$. Sledi
        \begin{equation*}
        \setlength{\jot}{10pt}
            \begin{aligned}
                x^{*} x &=  (\mathrm{Re} \: x - \mathrm{Im} \: x)(\mathrm{Re} \: x + \mathrm{Im} \: x)\\
                &= {(\mathrm{Re} \: x)}^2 + (\mathrm{Re} \: x)(\mathrm{Im} \: x) - (\mathrm{Im} \: x)(\mathrm{Re} \: x) - (\mathrm{Im} \: x)^2\\
                &= {(\mathrm{Re} \: x)}^2 - (\mathrm{Re} \: x)(\mathrm{Im} \: x) + (\mathrm{Im} \: x)(\mathrm{Re} \: x) - (\mathrm{Im} \: x)^2\\
                &= (\mathrm{Re} \: x + \mathrm{Im} \: x)(\mathrm{Re} \: x - \mathrm{Im} \: x)\\
                &= x x^{*}.
            \end{aligned}
        \end{equation*}
        Sedaj upoštevamo še zvezo $(\mathrm{Im} \: x)^2 = -{|\mathrm{Im} \: x|}^2$ iz dokaza točke \textit{ii}.\ in dobimo
        \begin{equation*}
        \setlength{\jot}{10pt}
            \begin{aligned}
                x^{*} x &=  (\mathrm{Re} \: x - \mathrm{Im} \: x)(\mathrm{Re} \: x + \mathrm{Im} \: x)\\
                &= {(\mathrm{Re} \: x)}^2 + \mathrm{Re} \: x \: \mathrm{Im} \: x - \mathrm{Re} \: x \: \mathrm{Im} \: x - (\mathrm{Im} \: x)^2\\
                &= {(\mathrm{Re} \: x)}^2 + {|\mathrm{Im} \: x|}^2\\
                &= {|x|}^2.
            \end{aligned}
        \end{equation*}
        \item Naj bosta $x = x_{0} + x_{1}i + x_{2}j + x_{3}k, \, y = y_{0} + y_{1}i + y_{2}j + y_{3}k \in \mathbb{H}$. Izračunajmo
        \begin{equation*}
        \setlength{\jot}{10pt}
            \begin{aligned}
                (xy)^{*} & = \Big((x_{0} + x_{1}i + x_{2}j + x_{3}k)(y_{0} + y_{1}i + y_{2}j + y_{3}k)\Big)^{*} \\
                & = (x_{0}y_{0} + x_{0}y_{1}i + x_{0}y_{2}j + x_{0}y_{3}k + x_{1}y_{0}i - x_{1}y_{1} + x_{1}y_{2}k - x_{1}y_{3}j \\
                &+ x_{2}y_{0}j - x_{2}y_{1}k - x_{2}y_{2} + x_{2}y_{3}i + x_{3}y_{0}k + x_{3}y_{1}j - x_{3}y_{2}i - x_{3}y_{3})^{*} \\
                &= ((x_{0}y_{0} - x_{1}y_{1} - x_{2}y_{2} - x_{3}y_{3}) + (x_{0}y_{1} + x_{1}y_{0} + x_{2}y_{3} - x_{3}y_{2})i \\
                &+ (x_{0}y_{2} - x_{1}y_{3} + x_{2}y_{0} + x_{3}y_{1})j + (x_{0}y_{3} + x_{1}y_{2} - x_{2}y_{1} + x_{3}y_{0})k)^{*} \\
                &= (x_{0}y_{0} - x_{1}y_{1} - x_{2}y_{2} - x_{3}y_{3}) - (x_{0}y_{1} + x_{1}y_{0} + x_{2}y_{3} - x_{3}y_{2})i \\
                &- (x_{0}y_{2} - x_{1}y_{3} + x_{2}y_{0} + x_{3}y_{1})j - (x_{0}y_{3} + x_{1}y_{2} - x_{2}y_{1} + x_{3}y_{0})k \\
                &= (y_{0} - y_{1}i - y_{2}j - y_{3}k)(x_{0} - x_{1}i - x_{2}j - x_{3}k) \\
                &= y^{*}x^{*}.
            \end{aligned}
        \end{equation*}
        \item Pišimo $x = \mathrm{Re} \: x + \mathrm{Im} \: x$ in ${x}^* = \mathrm{Re} \: x - \mathrm{Im} \: x$. Sledi
        \begin{equation*}
            \setlength{\jot}{10pt}
                \begin{aligned}
                    x = x^{*} &\Longleftrightarrow \mathrm{Re} \: x + \mathrm{Im} \: x = \mathrm{Re} \: x - \mathrm{Im} \: x \\
                    &\Longleftrightarrow \mathrm{Im} \: x = 0 \\
                    &\Longleftrightarrow x \in \mathbb{R}.
                \end{aligned}
        \end{equation*}
        \item Naj bo $x = x_{0} + x_{1}i + x_{2}j + x_{3}k$. Iščemo rešitve enačbe $x^2 = -1$. Ta enačba je po točki \textit{iii.}\@ ekvivalentna enačbi
        $$(\mathrm{Re} \: x)^2 - |\mathrm{Im} \: x|^2 + 2 \, \mathrm{Re} \: x \, \mathrm{Im} \: x = -1.$$
        Če primerjamo imaginarni del leve in desne strani enačbe, dobimo
        $$\mathrm{Re} \: x \, \mathrm{Im} \: x = 0.$$
        Če bi bil $\mathrm{Re} \: x$ neničeln, bi bil $\mathrm{Im} \: x = 0$, kar je v protislovju z začetno enačbo.
        Torej je $\mathrm{Re} \: x = 0$. Zgornja enačba se poenostavi v 
        $$|\mathrm{Im} \: x| = 1.$$
        V množici rešitev enačbe so torej vsi kvaternioni $x = x_{1}i + x_{2}j + x_{3}k$, za katere je 
        $$\mathrm{Im} \: x = x_{1}^2 + x_{2}^2 + x_{3}^2 = 1.$$
        Ta enačba parametrizira enotsko sfero $S^2$ v $\mathbb{R}^{3}$, na kateri leži nešteto mnogo točk, 
        torej ima začetna enačba res nešteto mnogo rešitev v $\mathbb{H}$.
        \item Naj bo
        $x = x_0 + x_1 i + x_2 j + x_3 k$.
        Ker velja $k=ij$, lahko člen s $k$ zapišemo kot $x_3(ij)=(x_3 i)j$. Tako dobimo
        \[
        x = (x_0 + x_1 i) + (x_2 + x_3 i)j.
        \]
        Tu sta $x_0+x_1 i\in\mathbb{C}$ in $x_2+x_3 i\in\mathbb{C}$, zato lahko definiramo kompleksni števili $c:=x_0+x_1 i, \, d:=x_2+x_3 i $
        in pišemo $x=c+dj$. Pokazati moramo še, da je tak zapis enoličen. Če je 
        $c+dj = \tilde{c} + \tilde{d}j$
        še za neka $\tilde{c}, \, \tilde{d} \in\mathbb{C}$, potem je
        \[
        (c-\tilde{c})+(d-\tilde{d})j=0.
        \]
        Če pišemo $c - \tilde{c} = \mathrm{Re} \: (c-\tilde{c}) + \mathrm{Im} \: (c-\tilde{c}) \, i$ in $d-\tilde{d} = \mathrm{Re} \:(d-\tilde{d}) + \mathrm{Im} \: (d-\tilde{d})\, i$
        ter upoštevamo $ij = k$, iz zgornje enačbe dobimo
        $$\mathrm{Re} \: (c-\tilde{c}) + \mathrm{Im} \: (c-\tilde{c}) \, i + \mathrm{Re} \:(d-\tilde{d})\, j + \mathrm{Im} \: (d-\tilde{d})\, k = 0.$$
        Sledi $\mathrm{Re} \: (c-\tilde{c}) = \mathrm{Im} \: (c-\tilde{c}) = \mathrm{Re} \:(d-\tilde{d}) = \mathrm{Im} \:(d-\tilde{d})= 0$ oz.\ 
        $c-\tilde{c} = d-\tilde{d} = 0$. Tako dobimo $c=\tilde{c}$ in $d=\tilde{d}$.
        \item Pišimo $c = a + bi \in \mathbb{C}$. Računamo
        $$jc = j(a + bi) = ja + bji = aj - bk.$$
        Po drugi strani 
        $$\overline{c}j = (a -bi)j = aj - bij = aj - bk.$$
        \item Velja $|xy| = \sqrt{(xy)^{*}(xy)}$. Po točki \textit{iv}.\@ velja $|xy| = \sqrt{y^{*}x^{*}xy}$, to pa je po točki \textit{iii}.\@  enako $\sqrt{y^{*}{|x|}^2y} = |x|\sqrt{y^{*}y} = |x| |y|$.
    \end{enumerate}
\end{dokaz}

Opremljeni s temi lastnostmi se lahko prepričamo, da kvaternioni zadoščajo lastnostim obsega, kot omenjeno
v uvodu.

\begin{trditev}\label{obseg}
    Množica $(\mathbb{H}, +, \cdot)$ tvori obseg. Ničla v tem obsegu je enaka $0 = 0 + 0i + 0j + 0k$, enota pa $1 = 1 + 0i + 0j + 0k$.
\end{trditev}

\begin{dokaz}
    To, da $(\mathbb{H}, +)$ zadošča lastnostim Abelove grupe, je očitno. Prav tako je očiten obstoj enote za množenje. Oglejmo si dokaz asociativnosti množenja. Pokazati moramo, da za vse $x, y, z 
    \in \mathbb{H}$ velja 
    $(xy)z = x(yz)$.
    Pišimo $x = \mathrm{Re} \: x + \mathrm{Im} \: x$, $y = \mathrm{Re} \: y + \mathrm{Im} \: y$, $z = \mathrm{Re} \: z + \mathrm{Im} \: z$. Najprej zmnožimo $x$ in $y$:
        \begin{equation*}
            \setlength{\jot}{10pt}
                \begin{aligned}
                    xy &= (\mathrm{Re} \: x + \mathrm{Im} \: x)(\mathrm{Re} \: y + \mathrm{Im} \: y)\\
                    &= \mathrm{Re} \: x \:\mathrm{Re} \: y +  \mathrm{Re} \: x \:\mathrm{Im} \: y + \mathrm{Im} \: x \:\mathrm{Re} \: y + \mathrm{Im} \: x \:\mathrm{Im} \: y\\
                \end{aligned}
        \end{equation*}
        Sedaj ta izraz pomnožimo z desne z $z$:
        \begin{equation*} 
            \setlength{\jot}{10pt}
                \begin{aligned}
                    (xy)z &=  (\mathrm{Re} \: x \:\mathrm{Re} \: y +  \mathrm{Re} \: x \:\mathrm{Im} \: y + \mathrm{Im} \: x \:\mathrm{Re} \: y + \mathrm{Im} \: x \:\mathrm{Im} \: y)(\mathrm{Re} \: z + \mathrm{Im} \: z)\\
                    &= \mathrm{Re} \: x \:\mathrm{Re} \: y \: \mathrm{Re} \: z + \mathrm{Re} \: x \:\mathrm{Im} \: y \: \mathrm{Re} \: z + \mathrm{Im} \: x \:\mathrm{Re} \: y \: \mathrm{Re} \: z + \mathrm{Im} \: x \:\mathrm{Im} \: y \: \mathrm{Re} \: z\\
                    &\quad+ \mathrm{Re} \: x \:\mathrm{Re} \: y \: \mathrm{Im} \: z + \mathrm{Re} \: x \:\mathrm{Im} \: y \: \mathrm{Im} \: z + \mathrm{Im} \: x \:\mathrm{Re} \: y \: \mathrm{Im} \: z + \mathrm{Im} \: x \:\mathrm{Im} \: y \: \mathrm{Im} \: z\\
                \end{aligned}
        \end{equation*}
        Sedaj upoštevamo, da so $\mathrm{Re} \: x, \mathrm{Re} \: y, \mathrm{Re} \: z \in \mathbb{R}$, torej je po točki \textit{i}.\ trditve~\ref{eq1} zgornji izraz enak
        \begin{equation*}
            \setlength{\jot}{10pt}
                \begin{aligned}
                    (xy)z &=  \mathrm{Re} \: x \:\mathrm{Re} \: y \: \mathrm{Re} \: z + \mathrm{Re} \: x \:\mathrm{Im} \: y \: \mathrm{Re} \: z+ \mathrm{Re} \: x \:\mathrm{Re} \: y \: \mathrm{Im} \: z + \mathrm{Re} \: x \:\mathrm{Im} \: y \: \mathrm{Im} \: z\\
                    &\quad + \mathrm{Im} \: x \:\mathrm{Re} \: y \: \mathrm{Re} \: z+ \mathrm{Im} \: x \:\mathrm{Im} \: y \: \mathrm{Re} \: z+ \mathrm{Im} \: x \:\mathrm{Re} \: y \: \mathrm{Im} \: z+ \mathrm{Im} \: x \:\mathrm{Im} \: y \: \mathrm{Im} \: z\\
                    &= (\mathrm{Re} \: x + \mathrm{Im} \: x)(\mathrm{Re} \: y \: \mathrm{Re} \: z + \mathrm{Im} \: y \: \mathrm{Re} \: z + \mathrm{Re} \: y \: \mathrm{Im} \: z + \mathrm{Im} \: y \: \mathrm{Im} \: z) \\
                    &= x(yz).
                \end{aligned}
        \end{equation*}
        Sedaj pokažimo levo distributivnost. Če pokažemo, da velja $u(y + z) = uy + uz$ za $u \in \{i, j, k\}$, bo po definiciji množenja kvaternionov iz tega sledilo
        $x(y + z) = xy + xz$ za poljuben $x \in \mathbb{H}$. Naj bosta sedaj $y = y_{0} + y_{1}i + y_{2}j + y_{3}k$, $z = z_{0} + z_{1}i + z_{2}j + z_{3}k$ iz $\mathbb{H}$. Oglejmo si
        \begin{equation*}
            \setlength{\jot}{10pt}
                \begin{aligned}
                    i(y + z) &=  i((y_{0} + z_0) + (y_{1} + z_1)i + (y_{2} + z_{2})j + (y_{3}+ z_{3})k) \\
                             &= (y_{0} + z_0)i - (y_{1} + z_1) + (y_{2} + z_{2})k -(y_{3}+ z_{3})j \\
                             &= y_{0}i + z_0i - y_{1} - z_1 + y_{2}k + z_{2}k - y_{3}j - z_{3}j \\
                             &= iy + iz.
                \end{aligned}
        \end{equation*}
        Pri tem smo upoštevali definicijo seštevanja in množenja v $\mathbb{H}$, dejstvo, da $i$ komutira z elementi iz $\mathbb{R}$, in pravila množenja
        za $i, j, k$.
        Na enak način pokažemo še $j(y + z) = jy + jz$ in $k(y + z) = ky + kz$. Podobno pokažemo tudi desno distributivnost. 

        \medskip
        Za konec dokaza pokažimo še, da je vsak neničeln kvaternion obrnljiv. Točka \textit{iii}.\ trditve~\ref{eq1}
        pove, da velja
        $xx^{*} = x^{*}x = |x|^2  \geq 0.$
        Posebej, za $x\neq 0$ je $|x|^2 > 0$. Definiramo
        $
        x^{-1}\;:=\;\frac{x^{*}}{|x|^2}.
        $
        Potem velja
        \[
        xx^{-1} \;=\; x\,\frac{x^{*}}{|x|^2} \;=\; \frac{xx^{*}}{|x|^2} \;=\; \frac{|x|^2}{|x|^2} \;=\; 1.
        \]
        Ker velja $x^{*} x = x x^{*}$, res dobimo $xx^{-1} \;=\; x^{-1}x \;=\; 1$.
\end{dokaz}
Izkaže se, da obseg ni edina zanimiva algebraična struktura, ki jo tvorijo kvaternioni. Naslednja trditev pove, da lahko na kvaternione gledamo tudi kot na vektorski prostor.

\begin{trditev}\label{vektorski_prostor}
    Kvaternioni $\mathbb{H}$ so vektorski prostor nad $\mathbb{R}$ z bazo $\{1, i, j, k\}$. Ta vektorski prostor je izomorfen prostoru $\mathbb{R}^4$.
\end{trditev}

\begin{dokaz}
    Da $\mathbb{H}$ zadošča zahtevam za vektorski prostor, ni težko videti. Definirajmo preslikavo
    $\Phi : \mathbb{H} \rightarrow \mathbb{R}^4$ s predpisom
    $$\Phi(x_0 + x_1 i + x_2 j + x_3 k) = (x_0, x_1, x_2, x_3),$$
    kjer so $x_0, \, x_1, \, x_2, \, x_3 \, \in \mathbb{R}$. Inverz $\Phi ^{-1}: \mathbb{R}^4 \rightarrow \mathbb{H}$ te preslikave
    je enak 
    $$\Phi ^{-1} (x_0, x_1, x_2, x_3) = x_0 + x_1 i + x_2 j + x_3 k.$$
     Preveriti moramo, da za $\Phi$ velja 
    $$\Phi(\lambda x + \mu y) = \lambda \Phi(x) + \mu \Phi(y)$$
    za vse $x, y \in \mathbb{H}, \, \lambda, \mu \in \mathbb{R}$. Vzemimo poljubna $x = x_0 + x_1 i + x_2 j + x_3 k$ in $y = y_0 + y_1 i + y_2 j + y_3 k$
    iz $\mathbb{H}$ ter poljubna skalarja $\lambda, \mu \in \mathbb{R}$. Sledi
    \begin{equation*}
        \setlength{\jot}{10pt}
            \begin{aligned}
                \Phi(\lambda x + \mu y) &= \Phi(\lambda (x_0 + x_1 i + x_2 j + x_3 k) + \mu (y_0 + y_1 i + y_2 j + y_3 k)) \\
                                        &= \Phi(\lambda x_0 + \mu y_0 + (\lambda x_1 + \mu y_1) i + (\lambda x_2 + \mu y_2) j + (\lambda x_3 + \mu y_3) k) \\
                                        &= (\lambda x_0 + \mu y_0, \lambda x_1 + \mu y_1, \lambda x_2 + \mu y_2, \lambda x_3 + \mu y_3) \\
                                        &= \lambda (x_0, x_1, x_2, x_3) + \mu (y_0, y_1, y_2, y_3) \\
                                        &= \lambda \Phi(x) + \mu \Phi(y).
            \end{aligned}
    \end{equation*}
    Pri tem smo upoštevali, da $i, j, k$ komutirajo s skalarji iz $\mathbb{R}$.
\end{dokaz}

\begin{definicija}
    Kvaterniona $x$ in $y$ sta \textbf{podobna}, če obstaja $u \in \mathbb{H}\setminus \{0\}$, tako da velja
    $$u^{-1}xu = y.$$
    Pišemo $x \sim y$.
\end{definicija}

\begin{trditev}
    Relacija $\sim$ je ekvivalenčna relacija. 
\end{trditev}

\begin{dokaz}
    Dokazati želimo, da je $\sim$ ekvivalenčna relacija, torej da je refleksivna, simetrična in tranzitivna. Pokažimo najprej refleksivnost. Za vsak kvaternion $x \in \mathbb{H}$ velja
    $x = 1^{-1}\, x \, 1$,
    torej je $\sim$ refleksivna. Za dokaz simetričnosti moramo preveriti, ali iz $x \sim y$ sledi $y \sim x$. Recimo, da velja $x \sim y$. Obstaja torej
    $u \in \mathbb{H}\setminus \{0\}$, da je 
    $$u^{-1}xu = y.$$ 
    Če to enačbo z leve pomnožimo z $u$ in z desne z $u^{-1}$, dobimo 
    $$x = uyu^{-1} = v^{-1}yv$$
    za $v := u^{-1}$, torej velja tudi $y \sim x$. Ostane 
    nam dokazati še tranzitivnost. Naj bodo $x, z, y \in \mathbb{H}$ in recimo, da velja $x \sim y$ in $y \sim z$. To pomeni, da obstajata taka $u, v \in \mathbb{H}\setminus \{0\}$,
    da velja 
    $u^{-1}xu = y$
    in
    $v^{-1}yv = z$.
    Sedaj lahko zapišemo
    $$v^{-1}u^{-1}xuv = z.$$
    Iz dokaza trditve~\ref{obseg} vemo, da je $x^{-1}\;=\displaystyle \frac{x^{*}}{|x|^2}$, zato po točki \textit{iv}.\@ in \textit{ix}.\ trditve~\ref{eq1} sledi
    $$(uv)^{-1}xuv = z.$$
    Ker sta $u, v \neq 0$, sledi $uv \neq 0$ in res velja $x \sim z$, torej je $\sim$ tranzitivna relacija.
\end{dokaz}

\begin{opomba}
    Z $\left[x\right]$ označimo ekvivalenčni razred, ki pripada elementu $x$.
\end{opomba}

\begin{lema}\label{ekvivalencni_razred}
    Naj bo $a = a_0 + a_1 i  +  a_2 j  +  a_3 k  \in \mathbb{H}$. Označimo $\tilde{a} = a_0 \, + \, \sqrt{a_1^2 + a_2^2 + a_3^2} \, i$. Potem velja $a \sim \tilde{a}$, 
    torej $a \in \left[\tilde{a}\right]$.
\end{lema}

\begin{dokaz}
    Pokazati moramo, da obstaja $x \in \mathbb{H} \setminus \{0\}$, ki reši enačbo
    \begin{equation*}
        ax = x \tilde{a}.
    \end{equation*}
    Če je $a \in \mathbb{R}$, zgornjo enačbo reši $x = 1$. Za $a \in \mathbb{C}$ ločimo dve možnosti. Če je $a_1 > 0$, velja $a = \tilde{a}$ in enačbo reši $x = 1$. 
    Če pa je $a_1 < 0$, velja $\tilde{a} = \overline{a}$ in po točki \textit{viii}.\ trditve~\ref{eq1} enačbo reši $x = j$.
    Denimo sedaj, da $a \notin \mathbb{C}$. Označimo $r = \sqrt{a_1^2 + a_2^2 + a_3^2}$ in definiramo 
    $x = a_1 \, + \, r -a_3j + a_2k$. Sledi 
        \begin{equation*}
        \setlength{\jot}{10pt}
            \begin{aligned}
                ax &= (a_0 + a_1 i  +  a_2 j  +  a_3 k)( a_1  +  r -a_3j + a_2k)   \\
                &= a_0 \, a_1 + a_0 \, r - a_0 \, a_3 \, j + a_0 \, a_2 \, k + a_1^2 \, i  + a_1 \, r \, i - a_1\, a_3 \, k - a_1\, a_2 \, j \\
                &\quad + a_1\, a_2 \, j + a_2\, r \, j + a_2\, a_3 + a_2^2 \, i + a_1\, a_3 \, k + a_3\, r \, k + a_3^2 \, i - a_2\, a_3 \\
                &= r \, (a_1 \, i  +  a_2 \, j  +  a_3 \, k) + (a_1^2 + a_2^2 + a_3^2)\, i + a_0 \, (a_1 + r -a_3 \, j + a_2 \, k) \\
                &= ( a_1 \, + \, r -a_3j + a_2k) (a_0 \, + \, r \, i) \\
                &= x \tilde{a}.
            \end{aligned}
    \end{equation*}
    Iz $a \notin \mathbb{C}$ sledi $x \neq 0$, torej smo našli tak neničeln $x$, da velja $x^{-1}ax = \tilde{a}$.  
\end{dokaz}

\begin{lema}\label{pod}
    Kvaterniona $x$ in $y$ sta podobna natanko tedaj, ko velja $\mathrm{Re} \: x = \mathrm{Re} \: y$ in $|\mathrm{Im} \: x | = |\mathrm{Im} \: y |$.
\end{lema}

\begin{dokaz}
    Predpostavimo najprej, da za $x, y \in \mathbb{H}$ velja $\mathrm{Re} \: x = \mathrm{Re} \: y$ in $\lvert \mathrm{Im} \: x  \rvert = \lvert \mathrm{Im} \: y  \rvert$.
    Po lemi~\ref{ekvivalencni_razred} vemo, da velja  
    $x \sim  \mathrm{Re} \: x + \lvert \mathrm{Im} \: x  \rvert \, i = \mathrm{Re} \: y + \lvert \mathrm{Im} \: y  \rvert \, i \sim y$, torej sta $x$ in $y$ res podobna.
    
    \medskip
    Pokažimo, da velja tudi obratno. Recimo, da sta kvaterniona $x$ in $y$ podobna. Po lemi~\ref{ekvivalencni_razred} vemo, da velja  
    $x \sim  \mathrm{Re} \: x + \lvert \mathrm{Im} \: x  \rvert \, i$ in $y \sim  \mathrm{Re} \: y + \lvert \mathrm{Im} \: y  \rvert \, i$. Ker
    je $\sim$ tranzitivna relacija, sledi $\mathrm{Re} \: x + \lvert \mathrm{Im} \: x  \rvert \, i \sim \mathrm{Re} \: y + \lvert \mathrm{Im} \: y  \rvert \, i$.
    Pokazati moramo, da sta dve kompleksni števili z nenegativnima imaginarnima deloma podobni natanko tedaj, ko sta enaki. Vzemimo poljuben $x = a + bi \in \mathbb{C}$, in 
    poglejmo, katera kompleksna števila so v njegovem ekvivalenčnem razredu. Naj bo $q = q_0 + q_1 i + q_2 j + q_3 k \, \in \mathbb{H} \setminus \{0\}$. Po nekaj računanja dobimo
    \begin{equation*}
        \setlength{\jot}{10pt}
            \begin{aligned}
                q x q^{-1} &= q a q^{-1} + q bi q^{-1} = a + \frac{b}{|q|^2}qiq^*\\
                        &= a + \frac{1}{{\lvert q \rvert}^2} \, b((q_0^2 + q_1^2 - q_2^2 - q_3^2)i 
                        + 2(q_0 q_3 + q_1 q_2) j + 2(q_1 q_3 - q_0 q_2)k).   
            \end{aligned}
    \end{equation*}
    Nas zanimajo $q x q^{-1} \in \mathbb{C}$, torej mora za $q$ veljati
    $$q_0 q_3 + q_1 q_2 = 0, \quad q_1 q_3 - q_0 q_2 = 0.$$
    To je ekvivalentno enačbi $(q_1 + i q_0)(\,q_2 - i q_3\,) = 0$, torej mora veljati
    $q_2 = q_3 = 0$ ali $q_0 = q_1 = 0$.
    V obeh primerih sledi
    $$q x q^{-1} = \frac{1}{{\lvert q \rvert}^2} \, (\lvert q \rvert^2 a + \lvert q \rvert^2 bi) = a + bi.$$
    Iz tega pa dobimo, da je $\mathrm{Re} \: x = \mathrm{Re} \: y$ in $\lvert \mathrm{Im} \: x  \rvert = \lvert \mathrm{Im} \: y  \rvert$.
\end{dokaz}

\begin{opomba}\label{neskoncno_mnogo_el}
    Iz leme~\ref{pod} sledi, da za $x \notin \mathbb{R}$ ekvivalenčni razred $\left[x\right]$ vsebuje neskončno mnogo elementov, za $x \in \mathbb{R}$
    pa je $\left[x\right] = x$.
\end{opomba}

Za konec tega poglavja si oglejmo še izrek in lemo, ki bosta ključna za dokazovanje nekaterih trditev v nadaljevanju. 

\begin{izrek}\label{enacba}
    Naj bodo $\alpha, \beta, \gamma \in \mathbb{H}$. Če $\alpha$ in $\gamma$ nista podobna, ima enačba
    $$x\alpha - \gamma x = \beta$$
    enolično rešitev v $\mathbb{H}$.
\end{izrek}

\begin{dokaz}
    Naj bodo $\alpha, \beta, \gamma \in R$ in $\alpha \nsim \gamma$. Definiramo preslikavo $T: \mathbb{H} \rightarrow 
    \mathbb{H}$ s predpisom
    $$T(x) = x\alpha - \gamma x.$$
    Sledi, da je naša enačba ekvivalentna enačbi
    $T(x) = b$.
    Pokazati hočemo, da je ta preslikava bijektivna. Vemo, da za linearne preslikave velja, da so bijektivne natanko
    tedaj, ko imajo trivialno jedro. Če torej pokažemo, da je $T$ linearna preslikava, za katero velja $\ker T = \{0\}$,
    bomo s tem dokazali trditev. 
    
    \medskip
    Najprej se prepričajmo, da je $T$ res linearna preslikava. Trditev~\ref{vektorski_prostor} pove, da je $\mathbb{H}$ 
    vektorski prostor nad $\mathbb{R}$. Sedaj pokažimo, da je $T$ aditivna. Naj bosta $x, y \in \mathbb{H}$ in 
    $\lambda \in \mathbb{R}$.  Velja
    \begin{equation*}
        \setlength{\jot}{10pt}
            \begin{aligned}
                T(x + y) &= (x + y)\alpha - \gamma (x + y) \\
                &= x\alpha + y\alpha - \gamma x - \gamma y \\
                &= T(x) + T(y),
            \end{aligned}
    \end{equation*}
    torej je $T$ aditivna. Sedaj preverimo še, če je $T$ homogena. Oglejmo si
    \begin{equation*}
        \setlength{\jot}{10pt}
            \begin{aligned}
                T(\lambda x) &= (\lambda x)\alpha - \gamma (\lambda x) \\
                &= \lambda (x\alpha - \gamma x) \\
                &= \lambda T(x).
            \end{aligned}
    \end{equation*}
    Sledi, da je preslikava $T$ homogena, torej je res linearna. Sedaj si oglejmo jedro te preslikave. Recimo, da za nek
    $x \in \mathbb{H}$ velja $T(x) = 0$. Potem velja 
    $$x\alpha - \gamma x = 0$$
    oziroma
    $$x\alpha = \gamma x.$$
    Torej je $x$ v jedru natanko tedaj, ko je $x$ rešitev enačbe $x\alpha = \gamma x$. Recimo sedaj, da obstaja tak
    $x \neq 0$, ki reši to enačbo. Potem lahko obe strani z desne pomnožimo z $x^{-1}$ in dobimo
    $$\alpha = x^{-1}\gamma x.$$
    To pa pomeni, da sta $\alpha$ in $\gamma$ podobna, kar je v nasprotju s predpostavko. Torej je $x = 0$ edina rešitev enačbe
    $x\alpha = \gamma x$, oziroma $\ker T = \{0\}$. Iz tega pa že sledi, da je $T$ injektivna preslikava, torej ima enačba
    $x\alpha - \gamma x = \beta$
    res enlično rešitev v $\mathbb{H}$.
\end{dokaz}

\begin{opomba}
    Zgornji izrek velja tudi za splošne obsege. Dokaz lahko najdete v \textcite{johnson}.
\end{opomba}

\subsection{Alternativna definicija kvaternionov}

Obseg $\mathbb{H}$ lahko definiramo tudi na naslednji način.

\begin{definicija}
    Definiramo množico
    $$\mathbb{H}' = \Biggl\{
        \begin{bmatrix}
            a & b\\
            - \overline{b} & \overline{a}
        \end{bmatrix}
        ; \, a, b \in \mathbb{C}
        \Biggr\},$$
        ki je podmnožica $2 \times 2$ kompleksnih matrik, pri čemer z $\overline{x}$ označujemo konjugirano kompleksno število.
\end{definicija}
Da sta definiciji ekvivalentni, se lahko prepričamo z vpeljavo preslikave

\[ \mathscr{M}: \mathbb{H} \rightarrow \mathbb{H}', \]
\[  x = a + bj \longmapsto x' = 
    \begin{bmatrix}
        a & b\\
        - \overline{b} & \overline{a}
    \end{bmatrix}.
\]

\begin{trditev}
    Množica $\mathbb{H}'$ je obseg in preslikava $\mathscr{M}$ je izomorfizem obsegov $\mathbb{H}$ in $\mathbb{H}'$.
\end{trditev}

\begin{dokaz}
    Najprej pokažimo, da je preslikava $\mathscr{M}: \mathbb{H} \rightarrow \mathbb{H}'$ izomorfizem. 
    Najprej pokažemo, da ohranja seštevanje in množenje.
    Naj bosta $x = a + bj$ in $y = c + dj$ iz $\mathbb{H}$. Računamo
    \begin{equation*}\tag{1}
        \setlength{\jot}{10pt}
            \begin{aligned}
                \mathscr{M}(x + y) &= \mathscr{M}((a + c) + (b + d)j) \\
                &= \begin{bmatrix}
                    a + c & b + d\\
                    - \overline{(b + d)} & \overline{(a + c)}
                \end{bmatrix} \\
                &= \begin{bmatrix}
                    a & b\\
                    - \overline{b} & \overline{a}
                \end{bmatrix}
                +
                \begin{bmatrix}
                    c & d\\
                    - \overline{d} & \overline{c}
                \end{bmatrix}\\
                &= \mathscr{M}(a+bj) + \mathscr{M}(c+dj)\\
                &= \mathscr{M}(x) + \mathscr{M}(y),
            \end{aligned}
    \end{equation*} torej $\mathscr{M}$ ohranja seštevanje. Sedaj preverimo, če ohranja tudi množenje. Upoštevamo točko \textit{viii}.\ trditve~\ref{eq1} in dobimo
    \begin{equation*}\tag{2}
        \setlength{\jot}{10pt}
            \begin{aligned}
                \mathscr{M}(xy) &= \mathscr{M}((ac - b\overline{d}) + (ad + b\overline{c})j) \\
                &= \begin{bmatrix}
                    ac - b\overline{d} & ad + b\overline{c}\\
                    - \overline{(ad + b\overline{c})} & \overline{(ac - b\overline{d})}
                \end{bmatrix} \\
                &= \begin{bmatrix}
                    ac - b\overline{d} & ad + b\overline{c}\\
                    - \overline{ad} - \overline{b}c & {\overline{ac} - \overline{b}d}
                \end{bmatrix}\\
                &= \begin{bmatrix}
                    a & b\\
                    - \overline{b} & \overline{a}
                \end{bmatrix}
                \begin{bmatrix}
                    c & d\\
                    - \overline{d} & \overline{c}
                \end{bmatrix}\\
                &= \mathscr{M}(a+bj)\mathscr{M}(c+dj)\\
                &= \mathscr{M}(x)\mathscr{M}(y).
            \end{aligned}\\
    \end{equation*} Pri tem smo upoštevali, da za kompleksni števili $z,w$ velja $\overline{zw} = \overline{z}\,\overline{w}$.
    Velja tudi
    $$\mathscr{M}(1) = 
    \begin{bmatrix}
        1 & 0 \\
        0 & 1
    \end{bmatrix},$$
    torej $\mathscr{M}$ ohranja enoto. Sedaj pokažimo še, da je $\mathscr{M}$ bijektivna. Definiramo preslikavo \(\mathscr{N}:\mathbb{H}'\to\mathbb{H}\) s predpisom
    \[
    \mathscr{N}\!\left(
    \begin{bmatrix}
    a & b\\
    -\overline{b} & \overline{a}
    \end{bmatrix}
    \right)=a+bj.
    \]
    Ta preslikava je dobro definirana in velja
    \(\mathscr{N}\circ\mathscr{M}=\mathrm{id}_{\mathbb{H}}\) and 
    \(\mathscr{M}\circ\mathscr{N}=\mathrm{id}_{\mathbb{H}'}\), torej je \(\mathscr{M}\) bijektivna. Sledi, da je \(\mathscr{M}\) izomorfizem \(\mathbb{H} \cong \mathbb{H}'\).

    \medskip
    Sedaj pokažimo še, da je $\mathbb{H}'$ obseg. Iz računov (1) in (2) je razvidno, da je množica $\mathbb{H}'$ zaprta za seštevanje in množenje. Poleg tega
    velja
    $$\det 
    \begin{bmatrix}
        a & b\\
        - \overline{b} & \overline{a}
    \end{bmatrix}
    = |a|^2 + |b|^2 \geq 0.$$
    Ta izraz je enak 0 natanko takrat, ko je $a = b = 0$, torej ko je matrika enaka 0. Sledi, da je vsak neničeln element množice $\mathbb{H}'$ obrnljiv. 
    
\end{dokaz}

\section{Kvaternionske matrike}

V prostoru matrik nad splošnim obsegom nimamo teorije podobnosti, lastnih vrednostih, trikotnih form ipd.\@ Na polje lahko  
gledamo kot na \say{najbolj obsežno} algebrsko strukturo, v kateri študiramo lastne vrednosti itd.\@ V tem razdelku bomo poskusili
lastnosti matrik posplošiti na primer, ko obseg, nad katerim študiramo matrike, ni komutativen. 

\subsection{Osnovne definicije}

Najprej se spoznajmo z definicijo kvaternionske matrike.

\begin{definicija}
    Naj bo $M_{m \times n}(\mathbb{H})$ množica $m \times n$ matrik z elementi iz obsega kvaternionov. Elementom te 
    množice pravimo \textbf{kvaternionske matrike}.
\end{definicija}

\noindent
Na tej množici poleg navadnega seštevanja in matričnega množenja definiramo še množenje s skalarjem na sledeč način.

\begin{definicija}
    Naj bo $A = [a_{ij}] \in M_{m \times n}(\mathbb{H})$ in $q \in \mathbb{H}$. \textbf{Levo množenje s skalarjem} definiramo kot
    $$qA = 
    \begin{bmatrix}
        qa_{ij}
    \end{bmatrix}
    .$$
    Analogno definiramo \textbf{desno množenje s skalarjem} kot
    $$Aq = \begin{bmatrix}
        a_{ij}q
    \end{bmatrix}
    .$$
\end{definicija}

Ker je množenje v $\mathbb{H}$ asociativno in veljata distributivnostna zakona, je množenje kvaternionskih matrik asociativno. Podobno, ker je $1$ enota za množenje v $\mathbb{H}$,
je 
\[I_n =
\begin{bmatrix}
1 & 0  & \cdots & 0 \\
0 & 1  & \cdots & 0 \\
\vdots & \vdots & \ddots & \vdots \\
0 & 0  & \cdots & 1
\end{bmatrix}
\]
enota za množenje v $M_{n}(\mathbb{H})$. Enostavno je tudi videti, da za $A \in M_{m \times n}(\mathbb{H}), \, B \in M_{n \times o}(\mathbb{H})$ in $p, q \in \mathbb{H}$ velja
$$(qA)B = q(AB), \quad
(Aq)B = A(qB), \quad
(pq)A = p(qA).$$

\noindent
Opazimo, da je $M_{m \times n}(\mathbb{H})$ levi oz.\@ desni vektorski prostor nad $\mathbb{H}$, odvisno od definicije množenja s 
skalarjem. Na tem prostoru lahko izvajamo vse operacije kot na prostoru kompleksnih matrik, razen tistih, pri katerih je vključena komutativnost, npr.
$$(qA)B \neq A(qB)$$
v splošnem. Začnimo z nekaj osnovnimi definicijami, ki jih poznamo že iz kompleksnih matrik.

\begin{definicija}
    Naj bo $A =
        \begin{bmatrix}
            a_{ij}
        \end{bmatrix}
        \in M_{m \times n}(\mathbb{H})$. \textbf{Konjugirano matriko} matrike $A$ definiramo kot 
        \[ \overline{A} = 
        \begin{bmatrix}
            a_{ij}^{*}
        \end{bmatrix}, \]
        \textbf{transponirano matriko} matrike $A$ kot
        \[ A^{T} =
        \begin{bmatrix}
            a_{ji}
        \end{bmatrix} \]
        in \textbf{adjungirano matriko} matrike $A$  kot
        \[ A^{H} = (\overline{A})^{T}. \]
\end{definicija}
\begin{definicija}
    Naj bo $A =
        \begin{bmatrix}
            a_{ij}
        \end{bmatrix}
        \in M_{n}(\mathbb{H})$ kvadratna matrika. Če za $A$ velja
        $$AA^{H} = A^{H}A,$$
        pravimo, da je $A$ \textbf{normalna matrika}. Če zanjo velja
        $$A^{H} = A,$$
        pravimo, da je \textbf{hermitska matrika}. Če je
        $$A^{H}A = I,$$
        je $A$ \textbf{unitarna matrika}. Če obstaja tak $B \in M_{n}(\mathbb{H})$, da velja
        $$AB = BA = I,$$
        pravimo, da je matrika $A$ \textbf{obrnljiva}.
\end{definicija}

\subsection{Lastnosti kvaternionskih matrik}
Sedaj si oglejmo nekaj lastnosti kvaternionskih matrik.

\begin{trditev}\label{lastnosti_matrike}
    Naj bo $A\in M_{m \times n}(\mathbb{H})$  in $B\in M_{n \times p}(\mathbb{H})$. Veljajo naslednje izjave.
    \begin{enumerate}
        \item $(\overline{A})^{T} = \overline{(A^{T})}$.
        \item $(AB)^{H} = B^{H}A^{H}$.
        \item $(AB)^{-1} = B^{-1}A^{-1}$, če sta $A$ in $B$ obrnljivi.
        \item $(\overline{A})^{-1} \neq \overline{A^{-1}}$ v splošnem.
        \item ${(A^{T})}^{-1} \neq {(A^{-1})}^{T}$ v splošnem.
    \end{enumerate}
\end{trditev}

\begin{dokaz}
    Naj bosta $A = 
        \begin{bmatrix}
            a_{ij}
        \end{bmatrix}
        \in M_{m \times n}(\mathbb{H}),
        B = 
        \begin{bmatrix}
            b_{jk}
        \end{bmatrix}
        \in M_{n \times o}(\mathbb{H})$.

    \begin{enumerate}
        \item Računamo
        \begin{equation*}
            (\overline{A})^{T} = 
                \begin{bmatrix}
                    a_{ij}^{*}
                \end{bmatrix}^{T} 
                = 
                \begin{bmatrix}
                    {a_{ji}}^{*} 
                \end{bmatrix} \\
                = \overline{
                \begin{bmatrix}
                    a_{ji}
                \end{bmatrix}} \\
                = \overline{(A^{T})}.
        \end{equation*}

        \item Upoštevamo točko \textit{iv}.\@ trditve~\ref{eq1} in dobimo
        \[(AB)^{H} = \bigg[
                    \sum_{k = 1}^{n} b_{ik}^{*}a_{ji}^{*}
                    \bigg]
                    =
                    B^{H}A^{H}.
                    \]
        \item Pokazati moramo, da obstaja matrika $C \in M_{p}(\mathbb{H})$, da velja
        $$ABC = CAB = I,$$
        in da je $C = B^{-1}A^{-1}$.
        Upoštevamo, da je $I$ enota za množenje, in dobimo
        $$(AB)B^{-1}A^{-1} = A(BB^{-1})A^{-1} 
        = AIA^{-1} 
        = AA^{-1} 
        = I.$$
        Tudi po drugi strani
        $$B^{-1}A^{-1}(AB) = B^{-1}(A^{-1}A)B 
        = BIB^{-1} 
        = BB^{-1} 
        = I.$$
        \item Vzemimo matriko
    $A = \begin{bmatrix}
        i & k \\
        0 & j
    \end{bmatrix}$
    z inverzom
    $A^{-1} = \begin{bmatrix}
        -i & -1 \\
        0 & -j
    \end{bmatrix}$,
    in recimo, da zanjo velja
    $(\overline{A})^{-1} = \overline{A^{-1}}$.
    Potem je 
    $$I = \overline{A}(\overline{A})^{-1} 
    = \overline{A} \, \overline{{A}^{-1}} 
    =
    \begin{bmatrix}
        -i & -k \\
        0 & -j
    \end{bmatrix}
    \begin{bmatrix}
        i & -1 \\
        0 & j
    \end{bmatrix} 
    =
    \begin{bmatrix}
        1 & 2i \\
        0 & 1
    \end{bmatrix},$$
    kar je protislovje.
    \item Naredimo analog dokaza točke \textit{iv}.  
    \end{enumerate} 

\end{dokaz}



\subsection{Kompleksna adjungirana matrika}

Pri študiju kvaternionskih matrik se moramo ustaviti že pri zelo osnovnem vprašanju: Če poznamo desni inverz neke matrike, je to tudi njen levi inverz? 
Natančneje, ali za kvaternionski matriki $A, B$, za kateri velja $AB = I$, velja $BA = I$? Izkaže se, da odgovor na to vprašanje 
ni tako trivialen, kot bi sprva pričakovali. V pomoč pri reševanju tega problema nam je ideja, da kvaternionsko matriko
identificiramo s parom kompleksnih matrik. Ta princip je opisan v spodnji definiciji, pred tem pa potrebujemo naslednjo lemo.

\begin{lema}\label{razcep}
    Vsako kvaternionsko matriko $A \in M_{m \times n}(\mathbb{H})$ se da enolično zapisati kot 
    $$A = A_{1} + A_{2}j,$$
    kjer sta $A_{1}, A_{2} \in M_{m \times n}(\mathbb{C})$.
\end{lema}

\begin{dokaz}
    Naj bo $A = \big[a_{ij}\big] \in M_{m \times n}(\mathbb{H})$. Po točki \textit{vii}.\ trditve~\ref{eq1} vemo, da lahko $a_{ij} \in \mathbb{H}$ enolično zapišemo kot 
    $$a_{ij} = b_{ij} + c_{ij}j, \quad b_{ij}, \, c_{ij} \in \mathbb{C}.$$
    Sledi, da je
    $$\big[a_{ij}\big] = \big[b_{ij} + c_{ij}j\big] 
    = \big[b_{ij}\big] + \big[c_{ij}\big]j 
    = A_{1} + A_{2}j,$$
    kjer $A_{1}, A_{2} \in M_{m \times n}(\mathbb{C})$. 
    
    \medskip
    Pokažimo še, da je tak zapis enoličen. Recimo, da obstajata še taki $\tilde{A_{1}} = \big[\tilde{b}_{ij}\big], \, \tilde{A_{2}} = \big[\tilde{c}_{ij}\big] \in M_{m \times n}(\mathbb{C})$, 
    da velja $A = \tilde{A_{1}} + \tilde{A_{2}}j$. Potem bi lahko vsak element $A$ zapisali kot
    \[ a_{ij} = \tilde{b}_{ij} + \tilde{c}_{ij}j. \]
    Ker je tak zapis enoličen, sledi $\tilde{b}_{ij} = b_{ij}, \, \tilde{c}_{ij} = c_{ij}$
    za vse elemente $a_{ij}$ matrike $A$, torej je $\tilde{A_{1}} = A_{1}$ in $\tilde{A_{2}} = A_{2}$.
\end{dokaz}

\begin{definicija}
    Za matriko $A = A_{1} + A_{2}j \in M_{n} (\mathbb{H})$, kjer sta $A_{1}, A_{2} \in M_{n} (\mathbb{C})$, je $2n \times 2n$ matrika 
        $$\chi_{A} = 
        \begin{bmatrix}
            A_{1} & A_{2}\\
            - \overline{A_2} & \overline{A_1}
        \end{bmatrix} $$
        \textbf{kompleksna adjungirana matrika} matrike A.
\end{definicija}

\begin{opomba}
    Če je $A$ kompleksna matrika, je $A = A_{1}$, torej je njena adjungiranka enaka
    $$\chi_{A} =
    \begin{bmatrix}
        A & 0\\
        0 & \overline{A}
    \end{bmatrix}.
    $$
\end{opomba}

\begin{zgled}
    Oglejmo si matriko
    $
    P = 
    \begin{bmatrix}
        1 & x \\
        0 & 1
    \end{bmatrix}
    $,
    kjer je $x = x_{0} + x_{1}i + x_{2}j + x_{3}k = (x_{0} + x_{1}i) + (x_{2} + x_{3}i)j$. Potem je
    $$\chi_{P} = 
    \begin{bmatrix}
        1 & x_{0} + x_{1}i & 0 & x_{2} + x_{3}i\\
        0 & 1 & 0 & 0 \\
        0 & -x_{2} + x_{3}i & 1 & x_{0} - x_{1}i \\
        0 & 0 & 0 & 1
    \end{bmatrix}.
    $$
    
\end{zgled}

\noindent
Zgornja definicija zelo spominja na alternativno definicijo kvaternionov, pri kateri kvaternion identificiramo z $2\times 2$ 
kompleksno matriko. Izkaže se, da nas ta ideja vodi do odgovora na vprašanje o levem in desnem inverzu.
Naslednji izrek potrdi domnevo, da je sta v $M_{n}(\mathbb{H})$ levi in desni inverz matrike enaka.

\begin{izrek}\label{inverz}
    Naj bosta $A, B \in M_{n} (\mathbb{H})$. Če velja $AB = I$, potem je $BA = I$.
\end{izrek}

\begin{dokaz}
    Vemo, da trditev velja za kompleksne matrike.
    Zapišemo $A = A_{1} + A_{2}j$ in $B = B_{1} + B_{2}j$, kjer so $A_{1}, \, A_{2}, \, B_{1}, \, B_{2} \in M_{n} (\mathbb{C})$.
    Enakost $AB = I$ 
    lahko zapišemo kot
    \begin{equation*}
        \setlength{\jot}{10pt}
            \begin{aligned}
                I &=(A_{1} + A_{2}j)(B_{1} + B_{2}j) \\
                &= A_{1}B_{1} + A_{1}B_{2}j + A_{2}jB_{1} + A_{2}jB_{2}j \\
                &= (A_{1}B_{1} - A_{2}\overline{B_{2}})+(A_{1}B_{2} + A_{2}\overline{B_{1}})j.
            \end{aligned}  
    \end{equation*} 
    Pri tem smo upoštevali zvezo $j^2 = -1$ in točko \textit{viii}.\ trditve~\ref{eq1}.
    Sledi
    $$A_{1} \, B_{1} - A_{2} \, \overline{B_{2}} = I, \quad A_{1} \, B_{2} + A_{2} \, \overline{B_{1}} = 0.$$
    To lahko v matrični obliki zapišemo kot
        $$\big[ \begin{matrix}
            A_{1} & A_{2}
        \end{matrix} \big]
        \begin{bmatrix}
            B_{1} & B_{2}\\
            -\overline{B_{2}} & \overline{B_{1}}
        \end{bmatrix}
        = 
        \big[\begin{matrix}
            I & 0
        \end{matrix} \big].$$
    Sledi
    $$\chi_{A} \chi_{B} =
    \begin{bmatrix}
        A_{1} & A_{2} \\
        -\overline{A_{2}} & \overline{A_{1}}
    \end{bmatrix}
    \begin{bmatrix}
        B_{1} & B_{2}\\
        -\overline{B_{2}} & \overline{B_{1}}
    \end{bmatrix} 
    =
    \begin{bmatrix}
        I & 0\\
        -\overline{A_{2}} \, B_{1} - \overline{A_{1}} \, \overline{B_{2}} & -\overline{A_{2}} \, B_{2} + \overline{A_{1}} \, \overline{B_{1}}
    \end{bmatrix} 
    =
    \begin{bmatrix}
        I & 0\\
        0 & I
    \end{bmatrix}.$$
    Ker sta $\chi_{A}, \chi_{B}$ kompleksni matriki, velja 
    \begin{equation*}
        \setlength{\jot}{10pt}
            \begin{aligned}
                \chi_{B} \chi_{A} &=
                \begin{bmatrix}
                    B_{1} & B_{2}\\
                    -\overline{B_{2}} & \overline{B_{1}}
                \end{bmatrix}
                \begin{bmatrix}
                    A_{1} & A_{2} \\
                    -\overline{A_{2}} & \overline{A_{1}}
                \end{bmatrix} 
                =
                \begin{bmatrix}
                    I & 0\\
                    0 & I
                \end{bmatrix}.
            \end{aligned}  
    \end{equation*}
    Dobimo
    $$B_{1} \, A_{1} - B_{2} \, \overline{A_{2}} = I, \quad
                B_{1} \, A_{2} + B_{2} \, \overline{A_{1}} = 0,$$
    oziroma
    \begin{equation*}
        \setlength{\jot}{10pt}
            \begin{aligned}
                I &= (B_{1}A_{1} - B_{2}\overline{A_{2}})+(B_{1}A_{2} + B_{2}\overline{A_{1}})j \\
                &=(B_{1} + B_{2}j)(A_{1} + A_{2}j).
            \end{aligned}  
    \end{equation*} 
    Torej res velja $BA = I$.

\end{dokaz}


\begin{trditev}\label{eq5}
    Naj bosta $A, B \in M_{n}(\mathbb{H})$. Označimo z $I_{n}$ identiteto na prostoru $n \times n$ kvaternionskih matrik. Veljajo naslednje izjave.
    \begin{enumerate}
        \item $\chi_{I_{n}} = I_{2n}$.
        \item $\chi_{AB} = \chi_{A} \chi_{B}$.
        \item $\chi_{A+B} = \chi_{A} + \chi_{B}$.
        \item $\chi_{A^{-1}} = (\chi_{A})^{-1}$, če je $A$ obrnljiva.
        \item $\chi_{A^{H}} = (\chi_{A})^{H}$.
        \item $\chi_{A}$ je hermitska natanko tedaj, ko je $A$ hermitska.
        \item $\chi_{A}$ je unitarna natanko tedaj, ko je $A$ unitarna.
        \item $\chi_{A}$ je normalna natanko tedaj, ko je $A$ normalna.
    \end{enumerate}
\end{trditev}

\begin{dokaz}
    Naj bosta $A = A_{1} + A_{2}j, B = B_{1} + B_{2}j\in M_{n}(\mathbb{H})$, kjer so $A_{1}, A_{2}, B_{1}, B_{2} \in M_{n}(\mathbb{C})$.
    \begin{enumerate}
        \item Najprej zapišimo matriko $I_{n}$ kot
        $I_{n} = I_{n} + 0_{n}j$,
        pri čemer je $0_{n}$ $n \times n$ ničelna matrika. Sedaj lahko izračunamo
        $$\chi_{I_{n}} =
        \begin{bmatrix}
            I_{n} & 0_{n}\\
            -\overline{0_{n}} & \overline{I_{n}}
        \end{bmatrix}
        =
        \begin{bmatrix}
            I_{n} & 0_{n} \\
            0_{n} & I_{n}
        \end{bmatrix} 
        =I_{2n}.$$
        \item Po lemi \ref{razcep} lahko matriko $AB$ zapišemo kot
        \begin{equation*}
            \setlength{\jot}{10pt}
                \begin{aligned}
                    AB &= (A_{1} + A_{2}j)(B_{1} + B_{2}j) \\
                    &= (A_{1}B_{1} - A_{2}\overline{B_{2}})+(A_{1}B_{2} + A_{2}\overline{B_{1}})j.
                \end{aligned}  
        \end{equation*}
        Sedaj računamo 
        \begin{equation*}
            \setlength{\jot}{10pt}
                \begin{aligned}
                    \chi_{AB} &= 
                    \begin{bmatrix}
                        A_{1}B_{1} - A_{2}\overline{B_{2}} & A_{1}B_{2} + A_{2}\overline{B_{1}} \\
                        - \overline{(A_{1}B_{2} + A_{2}\overline{B_{1}})} & \overline{A_{1}B_{1} - A_{2}\overline{B_{2}}}
                    \end{bmatrix} \\
                    &=
                    \begin{bmatrix}
                        A_{1}B_{1} - A_{2}\overline{B_{2}} & A_{1}B_{2} + A_{2}\overline{B_{1}} \\
                        - \overline{A_{1}B_{2}} - \overline{A_{2}}B_{1} & \overline{A_{1}B_{1}} - \overline{A_{2}}B_{2}
                    \end{bmatrix} \\
                    &= 
                    \begin{bmatrix}
                        A_{1} & A_{2} \\
                        -\overline{A_{2}} & \overline{A_{1}}
                    \end{bmatrix} 
                    \begin{bmatrix}
                        B_{1} & B_{2}\\
                        -\overline{B_{2}} & \overline{B_{1}}
                    \end{bmatrix} \\
                    &= \chi_{A} \chi_{B}.
                \end{aligned}  
        \end{equation*}
        \item Podobno kot zgoraj zapišimo matriko $A + B$ kot
        \begin{equation*}
            \setlength{\jot}{10pt}
                \begin{aligned}
                    A + B &= (A_{1} + A_{2}j) + (B_{1} + B_{2}j) \\
                    &= (A_{1} + B_{1}) + (A_{2} + B_{2})j.
                \end{aligned}  
        \end{equation*}
        Sledi
        \begin{equation*}
            \setlength{\jot}{10pt}
                \begin{aligned}
                    \chi_{A+B} &= 
                    \begin{bmatrix}
                        A_{1} + B_{1} &  A_{2} + B_{2}\\
                        - (\overline{A_{2} + B_{2}}) & \overline{A_{1} + B_{1}}
                    \end{bmatrix} \\
                    &=
                    \begin{bmatrix}
                        A_{1} + B_{1} & A_{2} + B_{2} \\
                        - \overline{A_{2}} - \overline{B_{2}} & \overline{A_{1}} + \overline{B_{1}}
                    \end{bmatrix} \\
                    &= 
                    \begin{bmatrix}
                        A_{1} & A_{2} \\
                        -\overline{A_{2}} & \overline{A_{1}}
                    \end{bmatrix} 
                    +
                    \begin{bmatrix}
                        B_{1} & B_{2}\\
                        -\overline{B_{2}} & \overline{B_{1}}
                    \end{bmatrix} \\
                    &= \chi_{A} + \chi_{B}.
                \end{aligned}  
        \end{equation*}
        \item Naj bo $A$ obrnljiva. Ker velja
        \[
        A\,A^{-1}=I_n \quad\text{in}\quad A^{-1}A=I_n,
        \]
        uporabimo točki \textit{i}.\@ in \textit{ii}.\@ ter dobimo
        \[
        \chi_A\,\chi_{A^{-1}}
        =\chi_{A A^{-1}}
        =\chi_{I_n}
        =I_{2n},
        \]
        in
        \[
        \chi_{A^{-1}}\,\chi_A
        =\chi_{A^{-1}A}
        =\chi_{I_n}
        =I_{2n}.
        \]
        
        Torej je $\chi_{A^{-1}}$ hkrati levi in desni inverz matrike $\chi_A$, zato
        \[
        \chi_{A^{-1}}=(\chi_A)^{-1}.
        \]
        \item Najprej izračunajmo
        $$A^{H} = (A_{1} + A_{2}j)^{H} 
                    = (A_{1})^{H} + (A_{2}j)^{H}. $$
        Sedaj upoštevamo točko \textit{iv}.\ trditve~\ref{eq1}, da dobimo
        \begin{equation*}
            \setlength{\jot}{10pt}
                \begin{aligned}
                    A^{H} &= A_{1}^{H} + j^{*} A_{2}^{H} \\
                    &= A_{1}^{H} - j A_{2}^{H} \\
                    &= A_{1}^{H} - A_{2}^{T} j.
                \end{aligned}  
        \end{equation*}
        Zadnja enakost velja po točki \textit{viii}.\ trditve~\ref{eq1}. Sedaj lahko zapišemo
        $$\chi_{A^{H}} = 
        \Bigg[\begin{matrix}
            A_{1}^{H} &  - A_{2}^{T} \\
            \overline{A_{2}^{T}} & \overline{A_{1}^{H}}
        \end{matrix}\Bigg] \\
        = 
        \begin{bmatrix}
            A_{1}^{H} & - \overline{A_{2}}^{H} \\
            A_{2}^{H} & \overline{A_{1}}^{H}
        \end{bmatrix} \\
        =
        \Bigg[\begin{matrix}
            A_{1} & A_{2} \\
            - \overline{A_{2}} & \overline{A_{1}}
        \end{matrix}\Bigg]^{H} \\
        = (\chi_{A})^{H}.$$
        \item Recimo, da je $A$ hermitska, torej je
        $A = A^{H}$.
        Potem po točki \textit{v.} velja
        $$\chi_{A} = \chi_{A^{H}} = \chi_{A}^{H},$$
        torej je tudi $\chi_{A}$ hermitska. Še obratno, naj bo $\chi_{A}$ hermitska, torej
        $\chi_{A} = \chi_{A}^{H}$.
        To je po točki \textit{v.} enako
        $$\chi_{A} = \chi_{A^{H}}.$$
        Sledi
        $A^{H} = A$,
        torej je A hermitska.
        \item Analog dokaza točke \textit{vi}, uporabimo točki \textit{ii}.\ in \textit{v}.
        \item Analog dokaza točke \textit{vi}, uporabimo točko \textit{v}.
    \end{enumerate}
\end{dokaz}

\section{Lastne vrednosti kvaternionskih matrik}
V tem razdelku se bomo posvetili lastnim vrednostim kvaternionskih matrik. Ker je množenje kvaternionskih matrik nekomutativno,
moramo enačbi $Ax = \lambda x$ in $Ax = x \lambda$ obravnavati ločeno.

\begin{definicija}
    Naj bo $A \in M_{n} (\mathbb{H})$ in $\lambda \in \mathbb{H}$. Če za nek $x \in \mathbb{H}^{n}\setminus\{0\}$ velja
    \[ Ax = \lambda x, \]
    pravimo, da je $\lambda$ \textbf{leva lastna vrednost} matrike $A$. Če za nek $x \in \mathbb{H}^{n}\setminus\{0\}$ velja
    \[ Ax = x \lambda, \]
    pravimo, da je $\lambda$ \textbf{desna lastna vrednost} matrike $A$. Množico
    \[ \sigma_{\ell}(A) = \{\lambda \in \mathbb{H} \mid \exists \, x \in \mathbb{H}^{n}\setminus \{0\}: Ax = \lambda x \} \]
    imenujemo \textbf{levi spekter} matrike $A$, množico
    \[ \sigma_{r}(A) = \{\lambda \in \mathbb{H} \mid \exists \, x \in \mathbb{H}^{n}\setminus \{0\}: Ax = x \lambda \} \]
    pa imenujemo \textbf{desni spekter} matrike $A$.
\end{definicija}


\begin{zgled}
    \begin{enumerate}
        \item Za matriko
        $
        A =
        \begin{bmatrix}
            1 & 0\\
            0 & i
        \end{bmatrix}
        $ velja $\sigma_{\ell}(A) = \{1, i\}$ in $\sigma_{r}(A) = \{1\} \cup \left[i\right]$.
        Preverimo, da so to res leve lastne vrednosti za $A$.
        Naj bo $x = \begin{bmatrix}x_1 \\ x_2\end{bmatrix} \in \mathbb{H}^2$.
        Oglejmo si
        \[
        Ax = \begin{bmatrix}x_1 \\ i x_2\end{bmatrix}, \quad
        \lambda x = \begin{bmatrix}\lambda x_1 \\ \lambda x_2\end{bmatrix},
        \]
        torej velja
        $$x_1 = \lambda x_1, \quad
                    i x_2 = \lambda x_2.$$
        Zanima nas, za katere $\lambda \in \mathbb{H}$ obstaja $x \in \mathbb{H}^2 \setminus \{0\}$, da sta ti dve enačbi izpolnjeni.
        Iz prve enačbe dobimo $\lambda = 1$, iz druge pa $\lambda = i$.
        Sedaj preverimo še desne lastne vrednosti za $A$. Za $x = \begin{bmatrix}x_1 \\ x_2\end{bmatrix} \in \mathbb{H}^2$ velja
        \[
        Ax = \begin{bmatrix}x_1 \\ i x_2\end{bmatrix}, \quad
        x \lambda = \begin{bmatrix} x_1 \lambda\\  x_2 \lambda \end{bmatrix}.
        \]
        Sledi
        $$x_1 = x_1 \lambda, \quad
        ix_2 = x_2 \lambda.$$
        Spet iščemo take $\lambda \in \mathbb{H}$, za katere obstaja $x \in \mathbb{H}^2 \setminus \{0\}$, da sta ti dve enačbi izpolnjeni.
        Iz prve enačbe dobimo $\lambda = 1$. Če predpostavimo $x_2 \neq 0$, pa dobimo $\lambda = x_2^{-1}ix_2$. Desne lastne vrednosti matrike $A$
        so torej $1$ in $[i]$.
        

        \item Matrika
        $A =
        \begin{bmatrix}
            0 & i\\
            j & 0
        \end{bmatrix}
        $
        ima lastne vrednosti $\sigma_{\ell}(A) = \{\frac{1}{\sqrt{2}}(i + j), - \frac{1}{\sqrt{2}}(i + j)\}$ in $\sigma_{r}(A) = \{\lambda \in \mathbb{H}: \: \lambda^4 + 1 = 0\}$.
        Preverimo najprej, da so to res leve lastne vrednosti za $A$. Naj bo $x = \begin{bmatrix}x_1 \\ x_2\end{bmatrix} \in \mathbb{H}^2$.
        Izračunamo
        \[
        A x 
        =
        \begin{bmatrix}
        i x_2 \\ j x_1
        \end{bmatrix},
        \quad
        \lambda x =
        \begin{bmatrix}
        \lambda x_1 \\ \lambda x_2
        \end{bmatrix}.
        \]
        Sledi
        $$i x_2 = \lambda x_1, \quad
                j x_1 = \lambda x_2.$$
        Prvi izraz zapišemo kot
        $
        x_2 = -i \lambda x_1
        $
        in ga vstavimo v drugega:
        \[
        j x_1 = \lambda x_2 = \lambda(- i \lambda x_1) = -\lambda i \lambda x_1.
        \]
        Zanima nas, za katere $\lambda$ obstaja $x \neq 0$, da zgornji izraz velja. Tako dobimo pogoj
        \begin{equation}\tag{1}
            j = -\lambda i \lambda.
        \end{equation}
        To enačbo moramo rešiti za $\lambda \in \mathbb{H}$.  
        Enačbo (1) z desne pomnožimo z $i$, da dobimo
        \begin{equation*}
            \setlength{\jot}{10pt}
            \begin{aligned}
                0 &= 2ab, \\
                0 &= -a^2 + b^2 -c^2 -d^2, \\
                1 &= 2bc, \\
                0 &= 2bd.
            \end{aligned}
        \end{equation*}
        Rešitev tega sistema je $a = d = 0$, $b = c = \pm \frac{1}{\sqrt{2}}$, torej sta iskani lastni vrednosti enaki
        $\lambda = \pm \frac{1}{\sqrt{2}}(i + j)$.
        Poiščimo še desne lastne vrednosti za $A$.
        Naj bo $x=\begin{bmatrix}x_1\\x_2\end{bmatrix} \in \mathbb{H}^{2}$. Izračunamo
        \[
        A x 
        =
        \begin{bmatrix}
        i x_2 \\ j x_1
        \end{bmatrix},
        \quad
        x \lambda =
        \begin{bmatrix}
        x_1 \lambda \\ x_2 \lambda
        \end{bmatrix}
        \]
        in dobimo
        \[
        i x_2 = x_1\lambda,\quad j x_1 = x_2\lambda. 
        \]
        Prvi izraz lahko zapišemo kot
        $
        x_2 = i^{-1} x_1\lambda = -i x_1\lambda 
        $
        in ga vstavimo v drugi izraz, da dobimo
        \[
        j x_1 = x_2\lambda = (-i x_1\lambda)\lambda = -i x_1\lambda^2.
        \]
        Če predpostavimo $x_1 \neq 0$, to lahko zapišemo kot
        $$
        \lambda^2 = x_1^{-1}k\,x_1.
        $$
        Velja torej \(\lambda^2 \sim k\). Sledi
        \[
        \lambda^4 = \big(x_1^{-1}k x_1\big)^2 = x_1^{-1}k^2 x_1 = x_1^{-1}(-1)x_1 = -1.
        \]
        S tem smo pokazali, da vsaka desna lastna vrednost \(\lambda\) matrike $A$ zadošča pogoju \(\lambda^4+1=0\). Če pokažemo še, da je
        vsak $\lambda \in \mathbb{H}$, ki reši enačbo $\lambda^4+1=0$, desna lastna vrednost za $A$, smo s tem dobili vse desne lastne vrednosti
        matrike $A$. Denimo sedaj, da je $\lambda \in \mathbb{H}$ tak, da velja $\lambda^4+1=0$.
        Pišimo \(u:=\lambda^2\). Potem je \(u^2=-1\), torej je \(u\) enotski strogo imaginarni kvaternion, tj.\ element oblike
        $$u=bi+cj+dk, \quad \sqrt{b^2 + c^2 + d^2} = 1.$$
        Po lemi~\ref{pod} vemo, da so kvaternioni tega tipa podobni, torej del istega ekvivalenčnega razreda. Posebej to pomeni, da obstaja tak \(x_1 \in 
        \mathbb{H}\setminus\{0\}\), da velja
        \[
        x_1^{-1}k x_1 = u.
        \]
        Sedaj definiramo
        $
        x_2 := -i x_1\lambda.
        $
        Sledi
        \[
        i x_2 = i(-i x_1\lambda) = x_1\lambda
        \]
        in z upoštevanjem \(u=x_1^{-1}k x_1\) dobimo
        \[
        x_2\lambda = -i x_1\lambda^2 = -i x_1 u = -i x_1(x_1^{-1}k x_1) = -i k x_1 = j x_1.
        \]
        Iz tega pa sledi, da je \(x= \begin{bmatrix} x_1 \\ x_2 \end{bmatrix} \neq 0\) lastni vektor
        za \(\lambda\). Torej je vsak \(\lambda\), za katerega velja \(\lambda^4+1=0\), desna lastna vrednost za $A$.
        \item Naj bo
        $A =
            \begin{bmatrix}
                0 & i \\
                -i & 0
            \end{bmatrix}
        $.
        Potem je $\lambda = k$ leva lastna vrednost matrike $A$ z lastnim vektorjem $x= \begin{bmatrix} 1 \\ j \end{bmatrix}$. Velja namreč
        \begin{equation*}
            Ax =
            \begin{bmatrix}
                0 & i \\
                -i & 0
            \end{bmatrix}
            \begin{bmatrix}
                1 \\
                j
            \end{bmatrix}
            =
            \begin{bmatrix}
                k \\
                -i
            \end{bmatrix} 
            =
            k
            \begin{bmatrix}
                1 \\
                j
            \end{bmatrix}
            =
            \lambda x.
        \end{equation*}
        Recimo, da je $k$ tudi desna lastna vrednost matrike $A$. Potem obstaja tak 
        \(x= \begin{bmatrix} x_1 \\ x_2 \end{bmatrix} \neq 0\), da velja
        $Ax = xk$
        oziroma
        $$i x_2 = x_1 k, \quad
                -i x_1 = x_2 k.$$
        Iz tega sledi
        $$-i(-ix_2k) = -x_2 k = x_2 k,$$
        zato $x_2 = x_1 = 0$, kar je v protislovju s predpostavko. Velja torej, da je $k$ leva lastna vrednost matrike $A$, ne pa tudi desna.
    \end{enumerate}
\end{zgled}

Iz zgornjega zgleda vidimo, da iskanje levih in desnih lastnih vrednosti ni trivialna naloga. V splošnem nimamo povezave
med levimi in desnimi lastnimi vrednostmi, opazimo pa zanimivo lastnost desnih lastnih vrednosti. 

\begin{lema}\label{desne lastne vrednosti}
    Naj bo $\lambda \in \mathbb{H}$ desna lastna vrednost matrike $A$. Potem so 
    tudi vsi elementi v $[\lambda]$ desne lastne vrednosti za $A$.
\end{lema}

\begin{dokaz}
    Če velja
    $Ax = x \lambda$,  
    sledi 
    $$A(xq) = (Ax)q = x\lambda q = (xq)(q^{-1}\lambda q)$$
    za vsak $q \neq 0$, torej je tudi $q^{-1}\lambda q$ desna lastna vrednost.
\end{dokaz}

\begin{trditev}
    Če je $A \in M_{n}(\mathbb{R})$ realna matrika, potem leve in desne lastne vrednosti $A$ sovpadajo, tj.
    \[\sigma_{\ell}(A) = \sigma_{r}(A).\]
\end{trditev}

\begin{dokaz}
    Naj bo $\lambda$ leva lastna vrednost $A$, torej $Ax = \lambda x$ za nek $x \neq 0$. Za vsak $q \in \mathbb{H}$, $q \neq 0$, velja
    \[(qAq^{-1})qx = (q\lambda q^{-1})qx\]
    in
    \[Aqx = (q\lambda q^{-1})qx,\]
    saj je $A$ realna matrika. Po lemi~\ref{ekvivalencni_razred} vemo, da obstaja $q \in \mathbb{H}$ tak, da je $q\lambda q^{-1}$ kompleksno število. Pišimo
    $qx = y = y_{1} + y_{2}j, \;  y_{1}, y_{2} \in \mathbb{C}^{n}$.
    Sledi
    \[Aqx = A(y_{1} + y_{2}j) = Ay_{1} + Ay_{2}j.\]
    To lahko dalje pišemo kot
    \[(q\lambda q^{-1})(y_{1} + y_{2}j) = \underbrace{(q\lambda q^{-1})}_{\in \mathbb{C}} \underbrace{y_{1}}_{\in \mathbb{C}} + \underbrace{(q\lambda q^{-1})}_{\in \mathbb{C}}\underbrace{y_{2}}_{\in \mathbb{C}}j,\]
    torej je 
    $Ay_{1} = y_{1}(q\lambda q^{-1})$ in $Ay_{2} = y_{2}(q\lambda q^{-1})$.
    Iz tega sledi, da je $q\lambda q^{-1}$ desna lastna vrednost matrike $A$, po lemi~\ref{desne lastne vrednosti} pa velja, da je tudi
    $\lambda$ desna lastna vrednost $A$. Na podoben način pokažemo tudi, da je vsaka desna lastna 
    vrednost matrike $A$ hkrati tudi njena leva lastna vrednost.
\end{dokaz}

Iz leme~\ref{desne lastne vrednosti} in opombe~\ref{neskoncno_mnogo_el} sledi, da ima kvaternionska matrika $A$ končno mnogo desnih lastnih vrednosti natanko tedaj, ko so
vse njene desne lastne vrednosti realne. Naslednja lema opisuje odnos med lastnimi vrednostmi kvaternionske matrike 
in njene kompleksne adjungiranke.

\begin{lema}\label{lastne vrednosti konjugiranke}
    Naj bo $\lambda \in \mathbb{H}$ desna lastna vrednost matrike $A \in M_{n}(\mathbb{H})$. Potem sta $\mu = \mathrm{Re} \: \lambda \pm 
    |\mathrm{Im} \: \lambda| \, i$ lastni vrednosti matrike $\chi_A$.
\end{lema}

\begin{dokaz}
    Naj bosta $A \in M_{n}(\mathbb{H})$ in $x \in \mathbb{H}^{n}$. Pišimo $A = A_1 + A_2j$ in $x = x_1 + x_2j$, kjer so 
    $A_1, A_2 \in M_{n}(\mathbb{C}), \: x_1, x_2 \in \mathbb{C}^n$. Naj bo $\lambda \in \mathbb{H}$ desna lastna vrednost
    za $A$ z lastnim vektorjem $x$. Iz leme~\ref{desne lastne vrednosti} vemo, da so potem tudi vsi elementi iz $[\lambda]$ desne lastne vrednosti
    za $A$. Naj bo $\mu$ kompleksno število iz $[\lambda]$.
    Enačbo $Ax = x\mu$ lahko zapišemo kot
    $$(A_1 + A_2j)(x_1 + x_2j) = (x_1 + x_2j)\mu,$$
    kar je enako
    $$A_1x_1 + A_1x_2j + A_2jx_1 + A_2jx_2j = x_1\mu + x_2j\mu.$$
    Sedaj upoštevamo zvezo $j^2 = -1$ in točko \textit{viii}.\ trditve \ref{eq1}, da dobimo
    $$(A_1x_1 - A_2\overline{x_2}) + (A_1 x_2 + A_2 \overline{x_1})j = x_1 \mu + x_2 \overline{\mu}j.$$
    Tako dobimo zvezi
    $$A_1x_1 - A_2\overline{x_2} = x_1 \mu, \quad
            A_1 x_2 + A_2 \overline{x_1} = x_2 \overline{\mu}.$$
    Naj bo sedaj $v = 
    \begin{bmatrix}
        x_1\\
        - \overline{x_2}
    \end{bmatrix}
    \in \mathbb{C}^2$. Računamo 
    \begin{equation*}
        \setlength{\jot}{10pt}
        \begin{aligned}
            \chi_A \, v =
            \begin{bmatrix}
                A_{1} & A_{2}\\
                - \overline{A_{2}} & \overline{A_{1}} 
            \end{bmatrix}
            \Bigg[\begin{matrix}
                x_1\\
                - \overline{x_2}
            \end{matrix}\Bigg]
            &= 
            \Bigg[\begin{matrix}
                A_1x_1 - A_2\overline{x_2}\\
                - \overline{A_2} x_1 - \overline{A_1} \overline{x_2}
            \end{matrix}\Bigg]
            &=
            \Bigg[\begin{matrix}
                A_1x_1 - A_2\overline{x_2}\\
                - \overline{(A_1 x_2 + A_2 \overline{x_1})}
            \end{matrix}\Bigg].
        \end{aligned}
    \end{equation*}
    Sedaj upoštevamo zgornji zvezi in dobimo
    \begin{equation*}
        \chi_A \, v =
        \begin{bmatrix}
            x_1 \mu \\
            - \overline{(x_2 \overline{\mu})}
        \end{bmatrix}
        =
        \begin{bmatrix}
            x_1 \\
            - \overline{x_2}
        \end{bmatrix}    
        \mu
        =\mu \, v,
    \end{equation*}
    torej je $\mu$ lastna vrednost za $\chi_A$. Dejstvo, da je $\mu = \mathrm{Re} \: \lambda \pm |\mathrm{Im} \: \lambda| \, i$, sledi 
    iz leme~\ref{pod}.
\end{dokaz}

\begin{opomba}\label{lastna vrednost adjungiranke}
    Iz zgornjega dokaza vidimo, da velja še več: če je $\lambda$ lastna vrednost za $\chi_A$,
    potem je $\lambda$ tudi desna lastna vrednost za $A$.
\end{opomba}

\begin{izrek}\label{zgornje trikotna}
    Naj bo $A \in M_{n}(\mathbb{H})$. Obstaja taka unitarna matrika $U \in M_{n}(\mathbb{H})$, da je matrika
    $U^{H}AU$ zgornje trikotna, tj.\ vsi elementi pod diagonalo so enaki 0.
\end{izrek}

Za dokaz tega izreka bomo potrebovali naslednji lemi.

\begin{lema}\label{nenicelna resitev}
    Naj bo $A \in M_{m \times n}(\mathbb{H})$, $m < n$. Potem ima enačba $Ax = 0$ neničelno rešitev.
\end{lema}

\begin{dokaz}
    Pišimo $A = A_1 + A_2j$ in $x = x_1 + x_2j$, kjer so 
    $A_1, A_2 \in M_{n}(\mathbb{C}), \: x_1, x_2 \in \mathbb{C}^n$. Enačbo $Ax = 0$ zapišemo v obliki
    $$\begin{bmatrix}
        A_{1} & A_{2}\\
        - \overline{A_{2}} & \overline{A_{1}} 
    \end{bmatrix}
    \Bigg[\begin{matrix}
        x_1\\
        - \overline{x_2}
    \end{matrix}\Bigg]
    = 0.$$
    Ker je $2m < 2n$, ima ta enačba neničelno rešitev.
\end{dokaz}

\begin{lema}\label{onb}
    Naj bo $u_1 \in \mathbb{H}^n$ enotski vektor. Obstajajo enotski vektorji $u_2, \ldots, u_n \in \mathbb{H}^n$, $n \geq 2$, da je
    $$\{u_1, u_2, \ldots, u_n\}$$
    ortonormirana množica, tj.\ velja
    $$u_i^* u_j = 
    \begin{cases}
        1 \: ; & i = j \\
        0 \: ; & i \neq j
    \end{cases}$$
    za vse $i, j = 1, \ldots, n$.
\end{lema}

\begin{dokaz}
    Naj bo $u_1 \in \mathbb{H}^n$ enotski vektor. Želimo najti enotski vektor $u_2 \in \mathbb{H}^n$, ki je pravokoten
    na $u_1$. Recimo, da je $A = u_i^*$ kvaternionska matrika velikosti $1 \times n$. Iščemo $x \neq 0$, da bo
    $$Ax = u_i^*x = 0.$$
    Ker je $A$ velikosti $1 \times n$ in velja $1 < n$, po lemi~\ref{nenicelna resitev} obstaja $x \neq 0$, ki reši
    to enačbo. Ta $x$ normiramo in dobimo $u_2$. Trditev torej velja za $n=2$.
    Denimo sedaj, da imamo ortonormirano množico $\{u_1, u_2, \ldots, u_k\}, \: k < n$. Sestavimo matriko $A$, ki naj ima za vrstice
    vektorje $u_1^*, u_2^*, \ldots, u_k^*$. Ta matrika je velikosti $k \times n$. Spet iščemo tak $x \in \mathbb{H}^{n}$,
    za katerega bo veljalo
    $$Ax = 0$$
    oziroma
    $$u_1^*x = 0, \:
    u_2^*x = 0, \:
    \ldots, \:
    u_k^*x = 0.$$
    Ker velja $k<n$, po lemi~\ref{nenicelna resitev} tak $x \neq 0$ obstaja. To rešitev normiramo in dobimo $u_{k+1}$. 
    Postopek ponavljamo, dokler ne dobimo vseh $n$ elementov ortonormirane množice.
\end{dokaz}

Sedaj se lahko lotimo dokaza izreka~\ref{zgornje trikotna}.

\begin{dokaz}
    Naj bo $A \in M_{n}(\mathbb{H})$. Dokaz poteka z indukcijo na $n$. Za $n = 1$ trditev očitno velja.
    Denimo sedaj, da trditev velja za matrike velikosti $(n-1)\times(n-1)$. Naj bo $u_1$ lastni vektor matrike $A$, 
    $\lVert u_1 \rVert = 1$, za desno lastno vrednost $\lambda_1$. Pri tem z $\lVert \cdot \rVert$ označujemo evklidsko 
    normo kvaternionskega vektorja. Lema~\ref{onb} zagotavlja, da obstajajo enotski vektorji $u_2, 
    \ldots, u_n$, da je $\{u_1, u_2,\ldots, u_n\}$ ortonormirana množica. Definiramo unitarno matriko $U_n$ kot
    \begin{equation*}
        U_n =
        \Bigg[\begin{matrix}
            u_1 & u_2 & \cdots & u_n
        \end{matrix}\Bigg]
        =
        \Bigg[\begin{matrix}
            u_1 & U_{n-1}
        \end{matrix}\Bigg].
    \end{equation*}
    Oglejmo si
    \begin{equation*}
        A \, U_n =
        \Bigg[\begin{matrix}
            A \, u_1 & A \, U_{n-1} 
        \end{matrix}\Bigg]
        =
        \Bigg[\begin{matrix}
            u_1\lambda_1 & A \, U_{n-1}
        \end{matrix}\Bigg].
    \end{equation*}
    Enačbo sedaj pomnožimo z leve z $U_n^H$ in dobimo
    \begin{equation*}
        U_n^H \, A \, U_n = 
        \Bigg[\begin{matrix}
           u_1^* \\
           U_{n-1}^H 
        \end{matrix}\Bigg]
        \Bigg[\begin{matrix}
            u_1\lambda_1 & A \, U_{n-1}
        \end{matrix}\Bigg]
        =
        \begin{bmatrix} 
            \lambda_1 & {u^*_1} \, A \,U_{n-1} \\
            \vdots & U_{n-1}^H \, A \, U_{n-1}   \\
            0      &                 &   \\ 
        \end{bmatrix}
        =
        \begin{bmatrix}
            \lambda_1 & \times \cdots \times \\
            \vdots & A_{n-1} \\
            0 &
        \end{bmatrix}.
    \end{equation*}
    Po indukcijski predpostavki obstaja $\widetilde{U}_{n-1} \in M_{n-1}(\mathbb{H})$ unitarna, da je matrika
    $$T := \widetilde{U}_{n-1}^H \, A_{n-1} \, \widetilde{U}_{n-1} $$ 
    zgornje trikotna. Če sedaj definiramo
    $$U = U_n
    \begin{bmatrix} 
        1      &\cdots           & 0 \\
        \vdots & \tilde{U}_{n-1} &   \\
        0      &                 &   \\ 
    \end{bmatrix},
    $$
    sledi
    \begin{equation*}
        U^H \, A \, U = 
        \begin{bmatrix}
            1 & \cdots &  0 \\
            \vdots  &\widetilde{U}^H_{n-1} &\\
            0 & &\\
        \end{bmatrix}
        \begin{bmatrix}
            \lambda_1 & \times \cdots \times \\
            \vdots & A_{n-1} \\
            0 &
        \end{bmatrix}
        \begin{bmatrix}
            1 & \cdots &  0 \\
            \vdots  &\widetilde{U}_{n-1} &\\
            0 & &\\
        \end{bmatrix}\\
        =
        \begin{bmatrix}
            \lambda_1 & \times \cdots \times \\
            \vdots & T \\
            0 &
        \end{bmatrix},
    \end{equation*}
   torej res velja, da je matrika $U^{H}AU$ zgornje trikotna.
\end{dokaz}

Naslednji izrek opisuje zanimivo lastnost kompleksnih lastnih vrednosti kvaternionske matrike.

\begin{trditev}\label{standardne}
    Vsaka matrika $A \in M_{n}(\mathbb{H})$ ima natanko $n$ desnih lastnih vrednosti oblike
    $$\mu = a + bi, \: b \geq 0.$$
\end{trditev}

\begin{dokaz}
    Naj bo $A \in M_{n}(\mathbb{H})$. Recimo, da je $\lambda \in \mathbb{H} \setminus \mathbb{R}$ desna lastna vrednost za 
    $A$. Če upoštevamo lemi~\ref{desne lastne vrednosti} in~\ref{pod}, sledi, da sta potem tudi $\mu = \mathrm{Re} \: \lambda + |\mathrm{Im} \: \lambda| \, i, \, \overline{\mu} = \mathrm{Re} \: \lambda - |\mathrm{Im} \: \lambda| \, i \in 
    \mathbb{C}$ desni lastni vrednosti za $A$. Za vsako desno lastno vrednost matrike $A$, ki ni iz $\mathbb{R}$, dobimo torej konjugiran par kompleksnih števil, ki
    je tudi v desnem spektru matrike $A$. Hkrati pa po lemi~\ref{lastne vrednosti konjugiranke} vemo, da so ti pari tudi lastne vrednosti
    za $\chi_A$. Ostane nam pokazati še, da imajo realne lastne vrednosti matrike $\chi_A$ sodo kratnost.

    \medskip
    To pokažemo z indukcijo na $n$. Za $n = 1$ trditev velja. Naj bo $n \geq 2$ in naj indukcijska predpostavka velja za matrike velikosti $2(n-1) \times 2(n-1)$. Pokazati hočemo, da 
    predpostavka velja za matrike velikosti $2n \times 2n$. Naj bo $a \in \mathbb{R}$ lastna vrednost za $A$, torej
    $Ax = ax = xa$ za nek enotski vektor $x\in \mathbb{H}^n \setminus \{0\}$. Po lemi~\ref{onb} vemo, da obstajajo $u_2, \ldots , u_n$, da je
    $$U =
    \Bigg[\begin{matrix}
        x & u_2 & \cdots & u_n
    \end{matrix}\Bigg]
    $$
    unitarna matrika. Označimo
    $$U^H A U = 
    \begin{bmatrix}
        a & \alpha ^T \\
        0 & B
    \end{bmatrix},
    $$
    pri čemer je $B = B_1 + B_2 j \in M_{n-1}(\mathbb{H})$ in $\alpha = \alpha_1 + \alpha_2 j \in \mathbb{H}^{n-1}$. Oglejmo si matriko
    \begin{equation*}\tag{1}
        \chi_{(U^H A U)} =  
        \begin{bmatrix}
            a & \alpha_1^T & 0 & \alpha_2^T \\
            0 & B_1 & 0 & B_2 \\
            0 & - \overline{\alpha_2}^T & a & \overline{\alpha_1}^T \\
            0 & - \overline{B_2} & 0 & \overline{B_1}
        \end{bmatrix}.
    \end{equation*}
    Hkrati po točkah \textit{ii}.\ in \textit{v}.\ trditve~\ref{eq5} velja $\chi_{(U^H A U)} = \chi_{U}^{H}\chi_A \chi_{U}$. Naj bo sedaj
    $$T =
    T =
    \begin{bmatrix}
    1 & 0 & \cdots & 0 \\
    0 & 0 & I_{n-1} & \vdots \\
    \vdots & I_{n-1} & 0 & 0 \\
    0 & \cdots & 0 & 1
    \end{bmatrix}$$
    permutacijska matrika. Potem velja 
    \begin{equation}\tag{2}
            T^{-1} \chi_{U}^{H}\chi_A \chi_{U} T =  
        \begin{bmatrix}
            a & 0 & \alpha_1^T & \alpha_2^T \\
            0 & a & - \overline{\alpha_2}^T & \overline{\alpha_1}^T\\
            0 & 0 & B_1 & B_2  \\
            0 & 0 & - \overline{B_2} & \overline{B_1}
        \end{bmatrix}
        =
        \begin{bmatrix}
            \chi_a & \chi_{\alpha^T} \\
            0 & \chi_B
        \end{bmatrix}.
    \end{equation}
    Matrika $T$ je unitarna. Po točki \textit{vii}.\ trditve~\ref{eq5} velja, da je tudi $\chi_U$ unitarna. Kompleksni matriki (1) in (2) sta torej podobni, torej imata enake lastne vrednosti. 
    Za $\chi_B$ po indukcijski predpostavki vemo, da ima vsaka njena realna lastna vrednost sodo algebraično
    večkratnost, zato ima lastna vrednost $a$ matrike $\chi_A$ sodo algebraično večkratnost.

    \medskip
    Ker je $\chi_A$ kompleksna matrika velikosti $2n \times 2n$, ima, štetih s kratnostjo, natanko $2n$ lastnih vrednosti. Iz tega sledi, da
    ima matrika $A$ natanko $n$ desnih lastnih vrednosti, ki so ali konjugirani pari kompleksnih števil ali realna števila s sodo algebraično večkratnostjo. Sledi, da natanko $n$ desnih lastnih vrednosti
    matrike $A$ leži na zgornji kompleksni polravnini (vključno z realno osjo).
\end{dokaz}

\begin{posledica}\label{sode lastne vrednosti}
    Naj bosta $A, B \in M_{n}(C)$. Potem ima vsaka realna lastna vrednost matrike
    $$
    \begin{bmatrix}
        A & B \\
        - \overline{B} & \overline{A}
    \end{bmatrix},
    $$
    (če obstaja), sodo algebraično večkratnost, in njene kompleksne lastne vrednosti se pojavljajo v konjugiranih parih.
\end{posledica}

\begin{opomba}
    Lastnim vrednostim iz trditve~\ref{standardne} pravimo \textbf{standardne lastne vrednosti} matrike $A$.
\end{opomba}

\begin{trditev}\label{podobne_diagonalnim}
    Naj bo $A \in M_{n}(\mathbb{H})$. Če je $A$ v trikotni obliki, tj.\ vsi elementi pod ali nad diagonalo so enaki 0,
    potem je vsak element na diagonali matrike $A$ desna lastna vrednost $A$. Še več, vsaka desna lastna 
    vrednost matrike $A$ je podobna kateremu od elementov na diagonali $A$.
\end{trditev}

\begin{dokaz}
    Najprej pokažimo prvi del trditve. Dokaz poteka z indukcijo $n$. Naj bo $A \in M_{n}(\mathbb{H})$ 
    zgornje trikotna z diagonalnimi elementi $\lambda_1, \lambda_2, \ldots, \lambda_n$. Oglejmo si najprej primer, ko je
    $n = 1$, torej $A = \lambda, \, \lambda \in \mathbb{H}$. Velja
    $$\lambda \, 1 = 1 \, \lambda,$$
    torej je $1$ lastni vektor za (desno) lastno vrednost $\lambda$. Naj bo sedaj $n \geq 2$.
    Predpostavimo sedaj, da trditev velja za matrike velikosti $n-1$. Matriko $A$ lahko zapišemo kot
    $$A =
    \begin{bmatrix}
        \lambda_1 & \alpha \\
        0 & A_{1}
    \end{bmatrix}.
    $$
    Velja
    $$A(1,0,\ldots, 0)^{T} = (1,0,\ldots, 0)^{T}\lambda_1,$$
    torej je $\lambda_1$ desna lastna vrednost za $A$. Po indukcijski predpostavki ima $A_{1}$ lastne vrednosti $\lambda_2, 
    \ldots, \lambda_n$.  Pokazati moramo le še, da so lastne vrednosti matrike $A_1$ tudi lastne vrednosti matrike $A$.
    Recimo, da je $\lambda \in \{\lambda_2, \ldots, \lambda_n\}$. Če je $\lambda$ podobna $\lambda_1$, velja
    $$\lambda = {q}^{-1}\lambda_1 q.$$
    Iz opombe~\ref{desne lastne vrednosti} vemo, da so vsi elementi iz $[\lambda_1]$ desne lastne vrednosti za $A$,
    torej je $\lambda$ desna lastna vrednost za $A$. 
    Predpostavimo torej lahko, da $\lambda$ in $\lambda_1$ nista 
    podobna. Velja $A_1 y = y \lambda$ za nek $y \neq 0$. Po trditvi~\ref{enacba} obstaja tak $x \in \mathbb{H}$, da velja
    $$\lambda_1 x + \alpha y = x\lambda.$$
    Sledi
    \begin{equation*}
        A
        \begin{bmatrix}
            x \\
            y
        \end{bmatrix}
        =
        \begin{bmatrix}
            \lambda_1 & \alpha \\
            0 & A_1
        \end{bmatrix}
        \begin{bmatrix}
            x \\
            y
        \end{bmatrix}
        =
        \begin{bmatrix}
            \lambda_1 x + \alpha y \\
            A_1 y
        \end{bmatrix}
        = 
        \begin{bmatrix}
            x \lambda \\
            y \lambda
        \end{bmatrix}
        =
        \begin{bmatrix}
            x \\
            y
        \end{bmatrix}
        \lambda,
    \end{equation*}
    torej je $\lambda$ res desna lastna vrednost za $A$. 

    \medskip
    Pokažimo še drugi del trditve. Lema~\ref{lastne vrednosti konjugiranke} 
    pove, da sta za poljubno desno lastno vrednost $\lambda$ matrike $A$ konjugirani kompleksni števili $\mu, \, \overline{\mu} \in [\lambda]$ lastni vrednosti za $\chi_A$. Ker so diagonalni elementi $A$ desne lastne vrednosti za $A$, sledi, da so jim lastne vrednosti $\chi_A$
    podobne. Iz tranzitivnosti relacije $\sim$ sledi, da je $\lambda$ podobna enemu od diagonalnih elementov matrike $A$.
    
\end{dokaz}

Oglejmo si naslednjo karakterizacijo obrnljivih matrik v $M_{n}(\mathbb{H})$.

\begin{trditev}\label{obrnljiva}
    Naj bo $A \in M_{n}(\mathbb{H})$. Naslednje izjave so ekvivalentne.
    \begin{enumerate}
        \item $A$ je obrnljiva.
        \item Enačba $Ax = 0$ ima enolično rešitev $x = 0$ v $\mathbb{H}^{n}$.
        \item $\chi_{A}$ je obrnljiva.
        \item $A$ nima ničelne lastne vrednosti, tj. če za nek $\lambda \in \mathbb{H}$ in $x \in \mathbb{H}^{n} \setminus \{0\}$ velja 
        $Ax = \lambda x$ ali $Ax = x \lambda$, potem je $\lambda \neq 0$.
    \end{enumerate} 
\end{trditev}

\begin{dokaz}
Naj bo $A \in M_{n}(\mathbb{H})$. 
\begin{itemize}
    \item (\textit{i}.\ $\Rightarrow$ \textit{ii}.) Trivialno.
    \item (\textit{ii}.\ $\Leftrightarrow$ \textit{iv}.) Izjava v točki \textit{ii}.\ pove, da ne obstaja $x \neq 0$ iz $\mathbb{H}^{n}$, da bi veljalo
    $$Ax = 0x = x0.$$
    To pa je ekvivalentno temu, da $A$ nima neničelne lastne vrednosti. 
    \item (\textit{iii}.\ $\Leftrightarrow$ \textit{iv}.) Matrika $\chi_{A}$ je obrnljiva natanko tedaj, ko nima ničelne lastne vrednosti. Iz dokaza 
    trditve~\ref{standardne} pa vidimo, da to velja natanko tedaj, ko $A$ nima ničelne lastne vrednosti.
    \item (\textit{iii}.\ $\Leftrightarrow$ \textit{i}.) Naj bo $\chi_{A}$ obrnljiva in naj bo
    \begin{equation*}
        \Bigg[\begin{matrix}
            B_{1} & B_{2} \\
            B_{3} & B_{4}
        \end{matrix}\Bigg]
        \begin{bmatrix}
            A_{1} & A_{2} \\
            - \overline{A_{2}} & \overline{A_{1}}
        \end{bmatrix}
        =
        \Bigg[\begin{matrix}
        I & 0 \\
        0 & I
        \end{matrix}\Bigg].
    \end{equation*}
    Zapišimo $B = B_{1} + B_{2}j$. Trivialno je preveriti, da velja $BA = I$ in po trditvi~\ref{inverz} sledi, da je 
    $A$ obrnljiva.
\end{itemize} 
\end{dokaz}

Za konec si oglejmo še trditev, ki s pomočjo do sedaj pridobljenih rezultatov karakterizira lastne vrednosti kvaternionskih matrik.

\begin{trditev}
    Naj bo $A \in M_n(\mathbb{H})$. Potem ima $A$ natanko $n$ desnih lastnih vrednosti iz različnih ekvivalenčnih razredov.
\end{trditev}

\begin{dokaz}
    Naj bo $A \in M_n(\mathbb{H})$. Iz trditve~\ref{standardne} vemo, da ima $A$ natanko $n$ standardnih lastnih vrednosti, tj.\ desnih lastnih
    vrednosti oblike $\lambda = a + bi, \: b \geq 0$. Recimo, da obstaja $\mu \in \mathbb{H}$,
    ki ne spada v noben ekvivalenčni razred teh standardnih lastnih vrednosti. Lema~\ref{ekvivalencni_razred} pove, da obstaja neko kopleksno število
    $z \in [\mu], \, \lvert \mathrm{Im} \: z\rvert \geq 0$, različno od ostalih standardnih lastnih vrednosti. Vendar bi po lemi~\ref{desne lastne vrednosti} 
    to pomenilo, da je tudi $z$ desna lastna vrednost za $A$, kar je v protislovju s predpostavko, da ima $A$ natanko $n$ standardnih lastnih vrednosti.
    Tak $\mu$ torej ne obstaja, zato vse desne lastne vrednosti spadajo v natanko $n$ ekvivalenčnih razredov.
\end{dokaz}

\printbibliography

\end{document}